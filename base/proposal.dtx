% \iffalse meta-comment
% A class for preparing general proposals, to be instantiated for particular donors
%
% Copyright (c) 2011 Michael Kohlhase, all rights reserved
%
% This file is distributed under the terms of the LaTeX Project Public
% License from CTAN archives in directory  macros/latex/base/lppl.txt.
% Either version 1.0 or, at your option, any later version.
%
% The development version of this file can be found at
% https://github.com/KWARC/LaTeX-proposal
% \fi
%
% \iffalse 
%<cls|sty|pdata|reporting>\NeedsTeXFormat{LaTeX2e}[1999/12/01]
%<cls>\ProvidesClass{proposal}[2016/12/15 v1.6 Project Proposal]
%<sty>\ProvidesPackage{proposal}[2016/12/15 v1.6 Project Proposal]
%<pdata>\ProvidesPackage{pdata}[2016/12/15 v1.6 Project Data]
%<reporting>\ProvidesPackage{reporting}[2016/12/15 v1.6 Project Reporting]
%<*driver>
\documentclass{ltxdoc}
\DoNotIndex{\def,\long,\edef,\xdef,\gdef,\let,\global}
\DoNotIndex{\begin,\AtEndDocument,\newcommand,\newcounter,\stepcounter}
\DoNotIndex{\immediate,\openout,\closeout,\message,\typeout}
\DoNotIndex{\section,\scshape,\arabic}
\EnableCrossrefs
%\CodelineIndex
%\OnlyDescription
\RecordChanges
\usepackage{textcomp,url,paralist,a4wide,xspace}
\usepackage[show]{ed}
\usepackage[maxbibnames=9,maxcitenames=2,hyperref=auto,style=alphabetic,backend=bibtex]{biblatex}
\addbibresource{../lib/proposal.bib}
% temporary fix due to http://tex.stackexchange.com/questions/311426/bibliography-error-use-of-blxbblverbaddi-doesnt-match-its-definition-ve
\makeatletter\def\blx@maxline{77}\makeatother
\usepackage[bookmarks=true,linkcolor=blue,
 citecolor=blue,urlcolor=blue,colorlinks=true,
 breaklinks=true, bookmarksopen=true]{hyperref}
\makeindex
\def\subversion{\textsc{Subversion}\xspace}
\def\git{\textsc{Git}\xspace}
\begin{document}
\DocInput{proposal.dtx}
\end{document}
%</driver>
% \fi
% 
%\iffalse\CheckSum{4091}\fi
% 
% \changes{v0.4}{2009/05/12}{copied over from eustrep.dtx}
% \changes{v0.5}{2009/05/27}{added gantt charts}
% \changes{v0.6}{2009/06/27}{added task referencing system}
% \changes{v0.7}{2010/09/22}{added final report infrastructure}
% \changes{v0.7}{2010/10/29}{added noRAM option}
% \changes{v0.7}{2010/11/01}{added paper list infrastructure}
% \changes{v1.0}{2010/11/03}{this works well, releasing to CTAN}
% \changes{v1.1}{2010/12/16}{deprecating workphase in favor of attributes}
% \changes{v1.1}{2011/01/05}{separating out pdata.sty for data handling and referencing}
% \changes{v1.2}{2011/03/18}{separating out dfgproposal.dtx}
% \changes{v1.2}{2011/03/19}{adding deliverables}
% \changes{v1.3}{2011/05/03}{adding general report handling}
% \changes{v1.4}{2012/01/15}{better metadata handling for task-level work
% phases and partners, using \texttt{wphases} instead of \texttt{start} and
% lists of sites}
% \changes{v1.4}{2012/01/18}{changing the argument structure of {\textbackslash joint*} to
% \texttt{len} in work packages and tasks}
% \changes{v1.4}{2012/01/18}{various extensions for the Jan 2012 EU proposals}
% \changes{v1.4}{2012/01/28}{some localization support}
% \changes{v1.4}{2013/01/12}{more documentation of milestones}
% \changes{v1.5}{2014/01/14}{many layout tweaks support for CVs, milestones in gantt}
% \changes{v1.6}{2016/10/26}{splitting off proposal.sty so that documentation becomes easier}
% \changes{v1.6}{2017/02/01}{adding site RM/RAM to tasks for fine-grained effort specification}
% 
% \GetFileInfo{proposal.cls}
% 
% \MakeShortVerb{\|} 
% \title{Preparing Proposals in {\LaTeX} with \texttt{proposal.cls}\thanks{Version {\fileversion} (last revised {\filedate})}}
% \author{Michael Kohlhase\\
%   Computer Science, Jacobs University Bremen\\
%   \url{http://kwarc.info/kohlhase}} 
% \maketitle
%
% \begin{abstract}
%   The |proposal| class supports many of the generic elements of Grant Proposals. It is
%   optimized towards collaborative projects, and should specialized to particular funding
%   agencies.
% \end{abstract}
% 
% \tableofcontents\newpage
%
% \section{Introduction}\label{sec:intro}
%
% Writing grant proposals is a collaborative effort that requires the integration of
% contributions from many individuals. The use of an ASCII-based format like {\LaTeX}
% allows to coordinate the process via a source code control system like \git or
% \subversion, allowing the proposal writing team to concentrate on the contents rather
% than the mechanics of wrangling with text fragments and revisions. In fact the
% |proposal| package has evolved out of a series of collaborative proposal writing
% efforts, where large teams (up to 30 individuals from up to 20 sites) have written a
% 100-page proposal in three weeks (with over 2000 commits). Such collaborative writing
% sprints are impossible without a revision control system and a ``semantic'' document
% class that generates tables, charts, and deliverable lists from content markup and thus
% takes care of many of the routine tasks of keeping information consistent.
% 
% The |proposal| class supports many of the generic elements of Grant Proposals. The
% package documentation is still preliminary, fragmented and incomplete. 
%
% The |proposal| class is distributed under the terms of the LaTeX Project Public License
% from CTAN archives in directory |macros/latex/base/lppl.txt|.  Either version 1.0 or, at
% your option, any later version. 
% 
% The CTAN archive always contains the latest stable version, the development version can
% be found on GitHub at \url{https://github.com/KWARC/LaTeX-proposal}. For bug reports
% please use the issue tracker there. Please feel free to fork the repository and provide
% extensions and improvements. 
% 
% The development version also contains example proposals and a very useful script that
% generates GitHub issues for all the workpackages, tasks, and deliverables. This is a
% great way of starting up a project and controlling its progress. The OpenDreamKit EU
% project (see \url{http://opendreamkit.org}) uses this for its (very public) project
% planning on the issue tracker at \url{https://github.com/OpenDreamKit} after (also
% publicly) developing the proposal on GitHub.
% 
% Finally, the GitHub reposistory contains example project proposals and specialized
% Makefiles that help start off the proposal development process. These are not part of
% the CTAN/TeXLive distributions.
%
% \section{The User Interface}\label{sec:user-interface}
% 
% In this section we will describe the functionality offered by the |proposal| class
% along the lines of the macros and environments the class provides. 
% 
% \subsection{Package Options}\label{sec:user:options}
%
% The |proposal| package takes the options |submit|, |noworkareas|, |RAM|, |deliverables|,
% |wpsubsection|, |keys|, |svninfo|, |gitinfo|, |numericcites|, and |public|.
% 
% The \DescribeMacro{submit}|submit| option will disable various proposal management decorations
% which are enabled by default for submission.
%
% The \DescribeMacro{noworkareas}|noworkareas| option specifies that we do not want to structure
% our work plan into work areas (see section~\ref{sec:user:wpwa}).
% 
% The \DescribeMacro{RAM}|RAM| option specifies that we specify research assistant months
% in the effort tallies (see section~\ref{sec:user:wpwa}).
% 
% The \DescribeMacro{deliverables}|deliverables| option specifies that we specify
% deliverables in the grant proposal (see section~\ref{sec:user:deliverables}). As the
% deliverables management needs extra support, we only activate them via this option.
% 
% The \DescribeMacro{wpsubsection}|wpsubsection| option specifies that we want to see
% subsections headings for the WPs (and WAs, if we have them).
% 
% The \DescribeMacro{report}|report| option specifies that we want to use the |report.cls|
% class as a basis for |proposal| instead of the default |article.cls|.
%
% The \DescribeMacro{keys}|keys| option specifies that we want to see the values of various
% keyval arguments in the margin.
% 
% The \DescribeMacro{svninfo}|svninfo| option specifies specifies that we want to use the
% |svninfo| package for displaying version control metadata in the document (except when
% the |submit| option is also given). For this we need the |svninfo| metadata line of the
% form 
% \begin{verbatim}
% \SVN $Id: proposal.tex 13610 2007-07-11 04:30:16Z kohlhase $
% \svnKeyword $HeadURL: https://svn.kwarc.info/../proposal.tex $
% \end{verbatim}
% at the beginning of each file (or in the preamble). 
% 
% Analogously, the \DescribeMacro{gitinfo}|gitinfo| option uses the |gitinfo2| package for
% GIT metadata. Note that you will need to install the post-commit hooks in your working
% copy according to~\cite{gitinfo2:on} for this to work.
% 
% The \DescribeMacro{numericcites}|numericcites| option changes citations to numeric from
% the default |alphabetic|.
%
% Finally, the \DescribeMacro{public}|public| option allows to hide certain sensitive
% (e.g. financial) parts of the proposal. For this, the |proposal| class provides the
% \DescribeEnv{private}|private| environment. If the option |public| is set, the parts of
% the document between |\begin{private}| and |\end{private}| do not produce output. This
% is useful for producing public versions of the proposal that hide confidential
% parts. Note that both |\begin{private}| and |\end{private}| {\emph{have to be on lines
% of their own may not have any leading whitespace}} otherwise an error occurs and
% {\LaTeX} gives error messages that are difficult to comprehend. An alternative way to
% distinguish private and public sections are to use the
% \DescribeMacro{\ifpublic}|\ifpublic| conditional: |\ifpublic{3}\else{5}\fi| will result
% in ``5'' in the submitted draft and ``3'' in the public document.
%
% \subsection{Proposal Metadata}\label{sec:user:metadata}
% 
% The metadata of the proposal is specified in the \DescribeEnv{proposal}|proposal|
% environment, which also generates the title page and the first section of the proposal
% as well as the last pages of the proposal with the signatures, enclosures, and
% references. The |proposal| environment should contain all the mandatory parts of the
% proposal text. The |proposal| environment uses the following keys to specify metadata.
% \begin{compactitem}
% \item \DescribeMacro{title}|title| for the proposal title (used on the title page),
% \item \DescribeMacro{instrument}|instrument| for the instrument of funding
%   that you would like to apply for,
% \item \DescribeMacro{acronym}|acronym| for the proposal acronym, possibly accompanied
%   by an \DescribeMacro{acrolong}|acrolong| that explains it. The acronym will also be used in
%   the page headings.
% \item \DescribeMacro{start}|start| for the start date of the proposed fragment of the
%   project, and \DescribeMacro{months}|months| for the length of the proposal in
%   months. Both have to be specified for the |proposal| class to work.
% \item If the proposal only concerns a part of a longer-running project, the
%   \DescribeMacro{since}|since| key allows to specify the date since when the overall
%   project runs. Finally, the \DescribeMacro{fundsuntil}|fundsuntil| allows to specify a
%   date until which the funds last.
% \item \DescribeMacro{discipline}|discipline| for the academic discipline and |areas| for
%   the research areas in that discipline.
% \item \DescribeMacro{PI}|PI| to declare the principal investigator. For collaborative
%   proposals we can use the |PI| key multiple times. The |proposal| package uses the
%   |workaddress| package for representation of personal metadata,
%   see~\cite{Kohlhase:workaddress:ctan} or the file |proposal.tex| for details.
% \item Many collaborative proposals are shared between two institutions, which we can
%   declare with the \DescribeMacro{site}|site| key. As this changes the interface this
%   should not be used for single-institution proposals. We will describe the setup for a
%   single-site proposal below and point out the differences. The example |proposal.tex|
%   is a two-site proposal.
% \end{compactitem}
% If the |acronym| and |acrolong| are given, then they automatically define the macros
% \DescribeMacro{\pn}|\pn| and \DescribeMacro{\pnlong}|\pnlong| which allow to use the
% project acronym (\underline{p}roject \underline{n}name) and its long version in the
% text. Note that these macros use |\xspace| internallly, so they do not have to be
% enclosed in curly braces.
% 
% There are two ways of organizing the distribution of personnel resources when developing
% a proposal. Either the coordinator takes a \emph{top-down approach} where she assigns
% person months (PM) to the respective site, of she takes a \emph{bottom-up approach},
% where the sites ``request'' personnel resources by marking them up in the CVs of the
% researchers in the site descriptions. |proposal.cls| supports both of these. Support
% for the first is configured via the \DescribeMacro{topdownPM}|topdownPM| key and for the
% other via the \DescribeMacro{botupPM}|botupPM| key. They add respective lines for
% planning in the WA/WP figure (see~\ref{sec:user:wpwa}).
% 
% \subsection{Proposal Appearance}\label{sec:user:appearance}
% 
% The |proposal| environment takes a second set of keyval arguments that allow to
% fine-tune the appearance of the proposal document. \ednote{move the RAM,
% wpsectionheadings,... options here.}
% \begin{compactitem}
% \item If the \DescribeMacro{compactht}|compactht| key is given (it does not need a
% value), then the header tables\ednote{describe them somewhere and reference here} are
% made compact, i.e. the sites that do not have a contribution to the work package or work
% area do not get listed. This is useful for proposals with more than 8 partners.
% \end{compactitem}
%
% The |proposal| package supplies the \DescribeEnv{emphbox}|emphbox| environment to create
% boxes of emphasized material we want to call attention to. 
% 
% \subsection{The proposal Environment and Title Page}\label{sec:user:title}
% \ednote{add documentation}
% 
% \subsection{Objectives}\label{sec:user:objectives}
% 
% The work plan starts with a discussion of objectives, which may be referenced in the
% text later. The |proposal| package provides the \DescribeEnv{objective}|objective|
% environment that allows to mark up individual objectives. It takes a keyval argument
% with the keys |id| for identification, |title| for the objective title, and |short| for
% a short title that can be used for referencing when the title is too long. The
% objectives can be referenced via \DescribeMacro{\OBJref}|\OJBref{|\meta{id}|}| by their
% label and via \DescribeMacro{\OBJtref}|\OJBtref{|\meta{id}|}| by label and (short if it
% was specified) title.
%
% \subsection{Work Areas and Work Packages}\label{sec:user:wpwa}
% 
% Grant proposals have another part that is often highly stylized; the work plan. This is
% usually structured into ``work packages'' --- i.e. work items that address a cohesive
% aspect of the proposed work. These work packages are usually consecutively numbered,
% have a title, and an associated effort estimation. As work packages are the ``atomic''
% planning units, they are usually heavily cross-referenced. A well-written proposal
% usually contains a table giving an overview over the work packages and their efforts and
% a Gantt chart showing the temporal distribution of the proposed work to allow the
% reviewers to get a clear picture of the feasibility of the research and development
% proposed. But this picture is also essential during the development of a proposal (which
% the |proposal| package aims to support), when the work packages (and their estimated
% efforts) usually change considerably. Therefore the |proposal| class standardizes
% markup for work packages and automatically computes the work package table (which can be
% inserted into the table via the \DescribeMacro{\wpfig}|\wpfig| macro) and the Gantt
% Chart (see Section~\ref{sec:user:gantt}).
%
% To achieve the automation, work plan is marked up by the
% \DescribeEnv{workplan}|workplan| environment, which sets up various internal counters and
% bookeeping macros. It contains texts and |workpackage| environments for the work
% packages.
%
% The purpose of the \DescribeEnv{workpackage}|workpackage| environment is to mark up a
% fragment of text as a work package description and specify the metadata so that it can
% be used in the work package table and Gantt chart generation. The metadata is specified
% by the following keys:
% \begin{compactitem}
% \item The \DescribeMacro{id}|id| key is used to specify a label for cross-referencing the
%   work package or work area, it must be document-unique.
% \item The \DescribeMacro{title}|title| and \DescribeMacro{short}|short| keys are used
%   for the work package/group title. The short title is used in tables and should not be
%   longer than 15 characters.
% \item The \DescribeMacro{wphases}|wphases| key is used according to
%   Section~\ref{sec:user:workphase}
% \item The \DescribeMacro{requires}|requires| key can be used to mark, up dependencies
%   between tasks. If |requires=\taskin{|\meta{rid}|}{|\meta{wp}|}| is given in a task
%   with |id=|\meta{t}, then task \meta{rid} in work package \meta{wp} must be completed
%   for task \meta{t} to become possible. This key will draw an arrow into the gantt chart
%   from the end of task \meta{rid} to \meta{t}. Note that dependencies should always
%   point forward in time. Furthermore, note that the fact that dependencies always go
%   from the end of the source to the beginning of the target work phase is intentional,
%   if this does not meet your needs, then you should probably break a work phase into
%   pieces that can be addressed separately.
% \item In single-site proposals, the \DescribeMacro{RM}|RM| (and \DescribeMacro{RAM}|RAM|
%   if the |RAM| option was given) keys are used to specify the estimated efforts to be
%   expended on research and development in this work package. Both are specified in
%   person months. |RM| is used for ``researcher months'' (wissenschaftlicher Mitarbeiter)
%   and |RAM| for ``research assistant months'' (wissenschaftliche Hilfskraft).
% \item In multi-site proposals, the |proposal| package generates the keys
%   \DescribeMacro{*RM}\meta{site}|RM| (and \DescribeMacro{*RAM}\meta{site}|RAM|) where
%   \meta{site} is any site label declared via the |site| key in the top-level |proposal|
%   environment. This can be used to specify the person months that the site spends on
%   this work package (the value for work areas is automatically computed (remember to
%   run {\LaTeX} twice for this)).
% \item In multi-site proposals the \DescribeMacro{lead} |lead| key specifies the work
%   package or work area lead, the value of this feature should be the short name of the
%   respective partner.
% \item For work packages with many prospers the \DescribeMacro{swsites}|swsites| key
%   can be given (no value needed) to turn the site names sideways to conserve
%   (horizontal) space. 
% \end{compactitem}
% 
% It is often useful to group the work packages in a proposal further (especially for
% larger, collaborative proposals). This can be done via the
% \DescribeEnv{workarea}|workarea| environment, which groups work packages. This
% environment takes the same keys as the |workpackage| environment, except for the
% efforts, which can be computed automatically from the work packages it groups.
% 
% As the author of the |proposal| class likes more structured proposals, using work
% areas is the default, but the |proposal| class can also be used with the
% |noworkareas| option for less structured (smaller) proposals.
% 
% \subsection{Tasks}\label{sec:user:task}
%
% In the work packages we can list tasks that need to be undertaken with the
% \DescribeEnv{tasklist}|tasklist| environment. The individual tasks are marked up with
% the \DescribeEnv{task}|task| environment.  This takes a keyval argument with the keys
% |id| for identification, |title| for a title, and the workphase keys |wphases|, |start|,
% |end|, and |force| (see Section~\ref{sec:user:workphase}). For planning involvement we
% can specify the overall person months via the |PM| key, the task lead via |lead|, and
% the partners involved via the |partners| key. Instead of just listing the partners, we
% can also specify the contributions of the partners with |RM|\meta{site} and
% |RAM|\meta{site} keys. Finally task dependencies can be specified via the |requires|
% key.
% 
% Tasks can be referenced by the \DescribeMacro{\taskref}|\taskref| macro that takes two
% arguments: the work package identifier and the task identifier. As for work packages and
% work areas, there is a long reference variant with work package title:
% \DescribeMacro{\tasktref}|\tasktref|. Finally,
% \DescribeMacro{\localtaskref}|\localtaskref| references a task in the local work package
% by the identifier in its argument.
% 
% \subsection{Work Phase Metadata}\label{sec:user:workphase}
% 
% The |task| and |workpackage| allow the \DescribeMacro{wphases}|wphases| key to specify
% the a list of work phases. The value of this key is comma-separated list of work phase
% specifications of the form \meta{start}|-|\meta{end} or
% \meta{start}|-|\meta{end}|!|\meta{force}, where \meta{start} and \meta{end} delimit the
% run time of the work phase and the optional |!|\meta{force} specifies the work force,
% i.e. the intensity of work as a number between 0 and 1. If no force is given, the
% default is 1. The main reason for specifying this metadata for tasks is to generate a
% Gantt chart (see Section~\ref{sec:user:gantt}).
%
% \subsection{Milestones and Deliverables}\label{sec:user:deliverables}
% 
% Many proposal formats foresee that project progress will be tracked in the form of
% \emph{milestones} -- points in the project, where a predefined state of affairs is
% reached -- and \emph{deliverables} -- tangible project outcomes that have to be
% delivered. Correspondingly, milestones and deliverables have to be specified in the
% proposal and accounted for in the project reports. To facilitate this the |proposal|
% class and its instances provide a simple infrastructure for dealing with milestones and
% deliverables. 
% 
% Milestones are usually given in a special table\footnote{this is the default provided by
% the base |proposal| class, it can be specialized for proposal class instances by
% redefining the |@milestones| environment and correspondingly the |\@milestone| macro.},
% which we markup up with the \DescribeMacro{milestones}|milestones| environment that
% takes care of initialization and numbering issues. This contains a list of milestone
% descriptions via the \DescribeMacro{\milestone}|\milestone| macro which is invoked as
% |\milestone[|\meta{keys}|]{|\meta{title}|}{|\meta{desc}|}|, where \meta{keys} supports
% the keys |id| for identification |month| for specifying the milestone date (in months of
% the project duration). Milestones are numbered with labels whose shape can be customized
% by redefining \DescribeMacro{\milestone@label}|\milestone@label| and referenced by the
% \DescribeMacro{\mileref}|\mileref{|\meta{id}|}| and
% \DescribeMacro{\miletref}|\miletref{|\meta{id}|}| for a reference with milestone
% title. |\pdatacount{all}{miles}| gives the number of milestones.
% 
% Deliverables are usually defined as part of the work package descriptions (see
% Section~\ref{sec:user:wpwa}) and listed in an overview table in a separate of the
% proposal. As for the milestones, we use an environment
% \DescribeMacro{wpdelivs}|wpdelivs| that contains the deliverable descriptions. These are
% marked up via the \DescribeMacro{wpdeliv} environment which takes an optional keyval
% argument for the deliverable metadata a regular argument for the title and contains the
% description of the deliverable as the body. For the metadata we have the keys |id| for
% the deliverable identifier, |due| for the target date (a number that denotes the project
% month), |nature| and |dissem| for specifying the deliverable nature and dissemination
% status (usually as short strings prescribed by the proposal template), and |miles| for
% the milestone this deliverable is targeted for (specified by the milestone
% identifier). For repeating deliverables (e.g. project reports), both |due| and |miles|
% can contain comma-separated lists. Deliverables are numbered by labels whose shape can
% be customized by number, where the shape of the label can be specified by redefining
% \DescribeMacro{\deliv@label}|\deliv@label| and referenced by
% \DescribeMacro{\delivref}|\delivref{|\meta{wp}|}{|\meta{id}|}| where \meta{wp} is the
% work package identifier and \meta{id} that if the deliverable and
% \DescribeMacro{\delivtref}|\delivtref{|\meta{wp}|}{|\meta{id}|}| for a reference with
% title. \DescribeMacro{\localdelivref}|\localdelivref| can be used to reference
% deliverables in the same work package. |\pdatacount{|\meta{wp}|}{delivs}| gives the
% number of milestones of the work package \meta{wp} |\pdatacount{all}{delivs}| that of
% all deliverables (aggregating over all work packages).
% 
% Some proposal templates ask for an overview table of the deliverables which aggregates
% the deliverables of the respective work packages and areas ordered by due date. This can
% be generated with the \DescribeMacro{\inputdelivs}|\inputdelivs| macro. This works index
% generation in {\LaTeX}. The |wpdeliv| environment writes the deliverable data to a file
% \meta{main}|.delivs|, which can be processed externally (usually just sorting with
% |sort| in Unix is sufficient) into \meta{main}|.deliverables|, which is then input via
% the |\inputdelivs| macro.
% 
% In some proposals, also work areas can have deliverables, then the above hold
% analogously for \DescribeMacro{wadelivs}|wpdelivs| and \DescribeMacro{wadeliv}|wadeliv|
% environments.
%
% Note that handling deliverables adds considerable overhead to proposal formatting and
% adds auxiliary files, so they are only activated if the |deliverables| option is given
% (see Section~\ref{sec:user:options}).
% 
% \subsection{Project Data,  Referencing, and Hyperlinking}\label{sec:user:hyperlinking} 
% 
% The |proposal| package extends the hyperlinking provided by the |hyperref| package it
% includes to work packages, work areas, \ldots. Whenever these are defined using the
% |proposal| infrastructure, the class saves the relevant information in the auxiliary
% file \meta{proposal}|.aux|. This information can be referenced via the
% \DescribeMacro  {\pdataref}|\pdataref| macro, which takes three arguments.
% 
% In a reference |\pdataref{|\meta{type}|}{|\meta{id}|}{|\meta{aspect}|}| the first argument
% \meta{type} specifies the type of the object (currently one of |wp|, |wa|, and
% |partner|) to be referenced, \meta{id} specifies the identifier of the referenced object
% (it matches the identifier given in the |id| key of the object), and \meta{aspect}
% specifies the aspect of the saved information that is referenced. 
%
% \DescribeMacro{\pdatarefFB}|\pdatarefFB{|\meta{type}|}{|\meta{id}|}{|\meta{a1}|}{|\meta{a2}|}|
% tries first |\pdataref{|\meta{type}|}{|\meta{id}|}{|\meta{a1}|}| and if that is not
% given |\pdataref{|\meta{type}|}{|\meta{id}|}{|\meta{a2}|}|. 
% 
% For a work package \meta{aspect} can be |number|, (the work package number), |label|
% (the label {\bf{WP$n$}} where $n$ is the work package number for referencing), |title|
% (the work package title), |lead| the work package leader, |short| (a short version of
% the WP title for tables). For work areas we have the same aspects with analogous
% meanings. In all cases, the referenced information carries a hyperlink to the referenced
% object.  
% 
% The \DescribeMacro{\pdataRef}|\pdataRef| and \DescribeMacro{\pdataRefFB}|\pdataRefFB|
% macros are variant of |\pdataref| and |\pdataRef| that also carry a hyperlink (if the
% |hyperref| package is loaded).
%
% The \DescribeMacro{\pdatacount}|\pdatacount| macro gives access to the numbers of
% certain aspects. For instance, the number of work packages in the proposal can be cited
% by |\pdatacount{all}{wp}|, similarly for work areas (if they are enabled), and finally,
% |\pdatacount{|\meta{wa}|}{wp}| gives the number of work packages for a work area
% \meta{wa}. This is very useful for talking about work plans in a general way. Other
% objects that can be counted are deliverables (|\pdatacount{all}{deliverables}|) and
% milestones (|\pdatacount{all}{milestones}|).
% 
% Note that since the referencable information is written into the project data file
% \meta{proposal}|.pdata| file, it is available for forward references. However, it will
% only become available when the project data file is read, so the proposal has to be
% formatted twice for references to be correct.
% 
% Finally, the |proposal| package supplies specialized reference macros for work packages
% and areas. The \DescribeMacro{\WPref}|\WPref| macro takes a work package identifier as
% an argument and makes a reference: |\WPref{|\meta{id}|}| abbreviates
% |\pdataRef{wp}{|\meta{id}|}{label}|. The \DescribeMacro{\WPtref}|\WPtref| macro is
% similar, but also prints out the (short) title: |\WPref{|\meta{id}|}| abbreviates
% |\pdataRef{wp}{|\meta{id}|}{label}: \pdataRef{wp}{|\meta{id}|}{title}|. Unless the
% |noworkareas| macro is set, we also have the variants \DescribeMacro{\WAref}|\WAref| and
% \DescribeMacro{\WAtref}|\WAtref| for work areas.
%
% \subsection{The Work Package Table}\label{sec:use:wptable}
% 
% One of the most useful features of the |proposal| class is that we can generate an
% overview table for the distribution of workloads in the project fully automatically. All
% it takes is the \DescribeMacro{\wpfig}|\wpfig| macro. We invoke this as
% |\wpfig[|\meta{opt}|]|, where \meta{opt} contains the following keywords: 
% 
% \begin{description}
% \item |pages| makes a column with page numbers of the respective work package/area description.
% \item |type| makes a column with work package/area types
% \item |start|, |end|, and |length| makes a columns with work package/area start/end
%   months and length (in months). 
% \item if |caption| is given then the table contains an explicatory caption.
% \item |label| allows to specify a label other than the default |fig:wplist|.
% \end{description}
% For instance |\wpfig[pages,start,length,caption=Overview of Work Packages]| gives a
% table with columns for page references, duration information, and a special caption.
% 
% The general appearance of the table \DescribeMacro{\wpfigstyle}|\wpfigstyle| macro takes
% a token sequence to specialize the global appearance (mostly used for text sizes and
% color) of the work package table. Cell styling can be tweaked by redefining special
% internal macros; see section~\ref{sec:implwptable}. 
%
% \subsection{Gantt Charts}\label{sec:user:gantt}
%
% Gantt charts are used in proposals to show the distribution of activities in work
% packages over time. A gantt chart is represented by the \DescribeEnv{gantt}|gantt|
% environment that takes a on optional keyval argument. The keys
% \DescribeMacro{xscale}|xscale| and \DescribeMacro{yscale}|yscale| are used to specify a
% scale factors for the chart so that it fits on the page. The \DescribeMacro{step}|step|
% key allows to specify the steps (in months) of the vertical auxiliary lines.  Finally,
% the \DescribeMacro{draft}|draft| key specifies that plausibility checks (that can be
% expensive to run) are carried out. Note that the value does not have to be given, so
% |\begin{gantt}{draft,yscale=.5,step=3}| is a perfectly good invocation.
%
% Usually, the |gantt| environment is not used however, since it is part of the
% \DescribeMacro{\ganttchart} macro that takes the same keys. This generates a whole
% Gantt chart automatically from the work phase specifications in the work packages. As
% above we have to run {\LaTeX} two times for the work phases to show up.
% 
% \subsection{Coherence}\label{sec:user:coherence}
% 
% Many proposals require ways to show coherence between the partners. The |proposal| class
% offers the macro \DescribeMacro{\coherencematrix}|\coherencematrix| for this which
% generates a matrix of symbols specifying joint publications, project organization,
% software/resource development, and supervision of students by the project partners that
% have been declared by the \DescribeMacro{\jointpub}|\jointpub|,
% \DescribeMacro{\jointsub}|\jointproj|, \DescribeMacro{\jointorga}|\jointorga|
% \DescribeMacro{\jointsoft}|\jointsoft|, and \DescribeMacro{\jointsup}|\jointsup| macros
% before. These macros all take a comma-separated list of site identifiers as an
% argument. Use for instance |\jointproj{a,b,c}| to specify that the sites with the
% identifiers |a|, |b| and |c| have a joint
% project. \DescribeMacro{\coherencetable}|\coherencetable| is a variant which packages
% the coherence table in a table figure with label |tab:collaboration|.
% 
% The symbols used an be configured by redefining \DescribeMacro{\jpub}|\jpub|,
% \DescribeMacro{\jproj}|\jproj|, and \DescribeMacro{\jorga}|\jorga|,
% \DescribeMacro{\jsoft}|\jsoft|, and \DescribeMacro{\jsup}|\jsup|.
%
% \subsection{Localization}\label{sec:use:localization}
%
% The |proposal| class offers some basic support for localization. This is still partial
% though, and I am not sure that this is the best way of setting things up. What I do is
% to define macros for all generated texts that can be redefined in the proposal classes
% that build in |proposal|. For instance the |dfgproposal| class~\cite{Kohlhase:pdrp:ctan}
% provides an option |german| for german-language proposals and project reports that
% triggers a redefinition of all of these macros at read time. 
%
% \section{Limitations and Enhancements}\label{sec:limitations}
% 
% The |proposal| is relatively early in its development, and many enhancements are
% conceivable. We will list them here.
% \begin{enumerate}
% \item macros cannot be used in work package and work area titles. They really mess up our
%   |\wpfig| automation. The problem is that they are evaluated too early, and our trick
%   with making them undefined while collecting the parts of the table-rows only works if
%   we know which macros we may expect. We might specify all ``allowable'' macros in an
%   optional key |protectmacro|, which is defined via\vspace*{-.7em}
% \begin{verbatim}
% \define@key{wpfig}{protectmacro}{\epandafter\let\csname #1\endcsname=\relax}
% \end{verbatim}
%  \vspace*{-2em} But I am not sure that this will work.
% \item It would be great, if in the Gantt Charts, we could include some plausibility
%   checks (for draft = not |submit| mode). I can see two at the moment:
%   \begin{itemize}
%   \item calculating the effort (i.e. the weight of the black area) and visualizing
%     it. Then we could check whether that is larger than the effort declared for the work
%     package.
%   \item calculating (and visualizing) the monthly effort. That should be kind of even
%     (or it has to be explained in the positions requested).
%   \end{itemize}
% \item we currently do not have a way to relate |PI|s to |site|s, but we do not really
%   need to.
% \end{enumerate}
% If you have other enhancements to propose or feel you can alleviate some limitation,
% please feel free to contact the author. 
%
% \section*{Acknowledgements}
%
% The author is indebted to Jake Hartenstein, Christoph Lange, Florian Rabe, Lutz
% Schr\"oder, and Tsanko Tsankov for error reports, feature suggestions, and code
% snippets.
% 
% \StopEventually{\newpage\PrintIndex\newpage\PrintChanges\newpage\printbibliography}\newpage
%
% \section{The Implementation}\label{sec:implementation}
%
% In this section we describe the implementation of the functionality of the |proposal| package. 
% 
% \subsection{Package Options and Format Initialization}\label{sec:impl:options}
% 
% We first set up the options for the package. 
% 
%    \begin{macrocode}
%<*cls|reporting>
\newif\if@wpsubsection\@wpsubsectionfalse
\newif\ifsubmit\submitfalse
\newif\ifgrantagreement\grantagreementfalse
\newif\ifpublic\publicfalse
\newif\ifkeys\keysfalse
\newif\ifdelivs\delivsfalse
\newif\ifwork@areas\work@areastrue
\newif\if@RAM\@RAMfalse
\newif\if@svninfo\@svninfofalse
\newif\if@gitinfo\@gitinfofalse
\newif\if@numericcites\@numericcitesfalse
\def\proposal@class{article}
\DeclareOption{wpsubsection}{\@wpsubsectiontrue}
\DeclareOption{submit}{\submittrue}
\DeclareOption{grantagreement}{\grantagreementtrue}
\DeclareOption{gitinfo}{\@gitinfotrue}
\DeclareOption{numericcites}{\@numericcitestrue}
\DeclareOption{svninfo}{\@svninfotrue}
\DeclareOption{public}{\publictrue}
\DeclareOption{noworkareas}{\work@areasfalse\PassOptionsToClass{\CurrentOption}{pdata}}
\DeclareOption{RAM}{\@RAMtrue}
\DeclareOption{report}{\def\proposal@class{report}}
\DeclareOption{keys}{\keystrue}
\DeclareOption{deliverables}{\delivstrue}
\DeclareOption*{\PassOptionsToClass{\CurrentOption}{article}}
\ProcessOptions
\LoadClass[a4paper,twoside]{\proposal@class}
\RequirePackage{proposal}
%</cls|reporting>
%    \end{macrocode}
% 
%    For |proposal.sty| we load the packages we make use of
% 
%    \begin{macrocode}
%<*sty>
\RequirePackage{amssymb}
\RequirePackage{wasysym}
\RequirePackage{url}
\RequirePackage{graphicx}
\RequirePackage{colortbl}
\RequirePackage{xcolor}
\RequirePackage{rotating}
\RequirePackage{fancyhdr}
\RequirePackage{array}
\RequirePackage{xspace}
\RequirePackage{comment}
\AtBeginDocument{\ifpublic\excludecomment{private}\fi}
\RequirePackage{tikz}
\RequirePackage{paralist}
\RequirePackage[a4paper,margin=18mm]{geometry}
\RequirePackage{boxedminipage}
% so that ednotes in wps do not run out of symbols
\renewcommand{\thempfootnote}{\roman{mpfootnote}}
\renewcommand{\familydefault}{\sfdefault}
\RequirePackage[scaled=.90]{helvet}
\RequirePackage{textcomp}
\if@numericcites
\RequirePackage[style=numeric,hyperref=auto,defernumbers=true,backend=bibtex,firstinits=true,maxbibnames=9,maxcitenames=3]{biblatex}[2010/11-19]
\else
\RequirePackage[style=alphabetic,hyperref=auto,defernumbers=true,backend=bibtex,firstinits=true,maxbibnames=9,maxcitenames=3]{biblatex}[2010/11-19]
\fi
\RequirePackage{csquotes}
\RequirePackage{mdframed}
%    \end{macrocode}
% in submit mode, we make the links a bit darker, so they print better. 
%    \begin{macrocode}
\RequirePackage{pdata}
\definecolor{darkblue}{rgb}{0,0,.7}
\ifsubmit\def\prop@link@color{darkblue}\else\def\prop@link@color{blue}\fi
\RequirePackage[bookmarks=true,linkcolor=\prop@link@color,
 citecolor=\prop@link@color,urlcolor=\prop@link@color,colorlinks=true,
 breaklinks=true, bookmarksopen=true]{hyperref}
%    \end{macrocode}
% 
% the |ed| package~\cite{Kohlhase:ed:ctan} is very useful for collaborative writing and
% passing messages between collaborators or simply reminding yourself of editing tasks, so
% we preload it in the class. However, we only want to show the information in draft
% mode. Furthermore, we adapt the options for the |svninfo| and |gitinfo2| packages.
%
%    \begin{macrocode}
\ifsubmit
\RequirePackage[hide]{ed}
\if@svninfo\RequirePackage[final,today]{svninfo}\fi
\else
\RequirePackage[show]{ed}
\if@svninfo\RequirePackage[eso-foot,today]{svninfo}\fi
\if@gitinfo\RequirePackage[mark]{gitinfo2}\fi
\fi
\renewcommand\ednoteshape{\sl\footnotesize}
%    \end{macrocode}
%    
% \begin{environment}{private}
%   We configure the |comment| package, so that it provides the |private| environment
%   depending on the status of the |public| option.
%    \begin{macrocode}
\ifpublic\excludecomment{private}\else\includecomment{private}\fi
%    \end{macrocode}
% \end{environment}
%
% And we set up the appearance of the proposal. We want numbered subsubsections.
%    \begin{macrocode}
\setcounter{secnumdepth}{3}
%    \end{macrocode}
% We specify the page headings. 
%    \begin{macrocode}
\let\prop@gen@acronym\@empty
\newif\ifofpage\ofpagefalse
\ifgrantagreement
\fancyhead{}
\renewcommand{\headrulewidth}{0pt}
\renewcommand{\footrulewidth}{0.4pt}
\else
\fancyhead[RE,LO]{\ifx\prop@gen@acronym\@empty\else\prop@gen@acronym\fi}
\fancyhfoffset{0pt}
\fi
\fancyfoot[C]{}
\newcommand\prop@of@pages[2]{page~#1\ifofpage~of~#2\fi}
\ifgrantagreement
\fancyfoot[L]{\prop@gen@proposalnumber%
\ifx\prop@gen@acronym\@empty\else\quad \prop@gen@acronym\fi\quad --\quad Part B}
\fancyfoot[R]{\thepage}
\else
\fancyhead[LE,RO]{\prop@of@pages\thepage{\pdataref@num{prop}{page}{last}}}
\fi
\pagestyle{fancyplain}
%</sty>
%    \end{macrocode}
%
% \subsection{Proposal Metadata}\label{sec:impl:metadata}
%  
% Most of the metadata functionality is encapsulated into the \DescribeMacro{pdata}|pdata|
% package, which is shared by the proposal and report classes. |pdata.sty| first loads the
% |workaddress| package from sTeX and supplies the Euro symbol. 
%    \begin{macrocode}
%<*pdata>
\RequirePackage{workaddress}[2016/07/06]
\RequirePackage{eurosym}
%    \end{macrocode}
% We define the keys for metadata declarations in the |proposal| environment, they park
% their argument in an internal macro for use in the title page. The |site| key is the
% most complicated, so we take care of it first: We need a switch |\if@sites| that is set
% to true when the |site| key is used. Furthermore |site=|\meta{site} makes new keys
% \meta{site}|RM| and \meta{site}|RAM| (if the |RAM| option was set) for the
% |workpackage| environment and records the sites in the |\prop@gen@sites| token register.
%    \begin{macrocode}
\newif\if@sites\@sitesfalse\let\prop@gen@sites=\relax%
\newcounter{@site}%
\define@key{prop@gen}{site}{\@sitestrue\@dmp{site=#1}%
\stepcounter{@site}\pdata@def{site}{#1}{number}{\the@site}%
\@ifundefined{prop@gen@sites}{\xdef\prop@gen@sites{#1}}{\xdef\prop@gen@sites{\prop@gen@sites,#1}}%
\define@key{prop@gen}{#1RM}{\pdata@def{site}{#1}{intendedRM}{##1}}%
\if@RAM\define@key{prop@gen}{#1RAM}{\pdata@def{site}{#1}{intendedRAM}{##1}}\fi
\define@key{workpackage}{#1RM}{\pdata@def\wp@id{#1}{RM}{##1}}%
\if@RAM\define@key{workpackage}{#1RAM}{\pdata@def\wp@id{#1}{RAM}{##1}}\fi
\define@key{task}{#1RM}{\pdata@def{\wp@id @\task@id}{#1}{RM}{##1}}%
\if@RAM\define@key{task}{#1RAM}{\pdata@def{\wp@id @\task@id}{#1}{RAM}{##1}}\fi
\define@key{prop@gen}{#1employed}{{\let\tabularnewline\relax\let\hline\relax\let\wa@ref\relax%
\@ifundefined{prop@gen@employed@lines}%
{\xdef\prop@gen@employed@lines{\wa@ref3{institution}{#1}{shortname} & ##1\tabularnewline\hline}}%
{\xdef\prop@gen@employed@lines{\prop@gen@employed@lines \wa@ref3{institution}{#1}{shortname} & ##1\tabularnewline\hline}}}}}
%    \end{macrocode}
% If there are no sites, then we have to define keys |RM| and |RAM| that store the
% intended research (assistant months). Unfortunately, we cannot just include this in the
% |\if@sites| conditional here, since that is only set at runtime. 
%    \begin{macrocode}
\define@key{prop@gen}{RM}{\@dmp{RM=#1}\if@sites%
\PackageWarning{Do not use the RM key in the presence of sites}\else%
\pdata@def{all}{intended}{RM}{#1}\fi}
\define@key{prop@gen}{RAM}{\@dmp{RAM=#1}\if@sites%
\PackageWarning{Do not use the RAM key in the presence of sites}\else%
\pdata@def{all}{intended}{RAM}{#1}\fi}
%    \end{macrocode}
%  similarly, the |PI| keys  are registered in |\prop@gen@PIs|.
%    \begin{macrocode}
\define@key{prop@gen}{PI}{\@dmp{PI=#1}%
\@ifundefined{prop@gen@PIs}{\xdef\prop@gen@PIs{#1}}{\xdef\prop@gen@PIs{\prop@gen@PIs,#1}}}
%    \end{macrocode}
%  and the |pubspage| keys  in |\prop@gen@pubspages|.
%    \begin{macrocode}
\define@key{prop@gen}{pubspage}{\@ifundefined{prop@gen@pubspages}%
{\xdef\prop@gen@pubspages{#1}}{\xdef\prop@gen@pubspages{\prop@gen@pubspages,#1}}}
%    \end{macrocode}
% the |importfrom| key reads the proposal data from its argument. 
%    \begin{macrocode}
\define@key{prop@gen}{importfrom}{\message{importing proposal data from #1.pdata}\readpdata{#1}}
%    \end{macrocode}
%  The rest of the keys just store their value. 
%    \begin{macrocode}
\define@key{prop@gen}{instrument}{\def\prop@gen@instrument{#1}%
\pdata@def{prop}{gen}{instrument}{#1}\@dmp{inst=#1}}
\define@key{prop@gen}{title}{\def\prop@gen@title{#1}%
\pdata@def{prop}{gen}{title}{#1}}
\define@key{prop@gen}{acronym}{\gdef\prop@gen@acronym{#1}%
\pdata@def{prop}{gen}{acronym}{#1}\@dmp{acro=#1}}
\define@key{prop@gen}{acrolong}{\def\prop@gen@acrolong{#1}%
\pdata@def{prop}{gen}{acrolong}{#1}}
\define@key{prop@gen}{proposalnumber}{\def\prop@gen@proposalnumber{#1}%
\pdata@def{prop}{gen}{proposalnumber}{#1}}
\define@key{prop@gen}{discipline}{\def\prop@gen@discipline{#1}%
\pdata@def{prop}{gen}{discipline}{#1}}
\define@key{prop@gen}{areas}{\def\prop@gen@areas{#1}%
\pdata@def{prop}{gen}{areas}{#1}}
\define@key{prop@gen}{start}{\def\prop@gen@start{#1}%
\pdata@def{prop}{gen}{start}{#1}}
\define@key{prop@gen}{months}{\def\prop@gen@months{#1}%
\pdata@def{prop}{gen}{months}{#1}}
\define@key{prop@gen}{since}{\def\prop@gen@since{#1}%
\pdata@def{prop}{gen}{since}{#1}}
\define@key{prop@gen}{totalduration}{\def\prop@gen@totalduration{#1}%
\pdata@def{prop}{gen}{totalduration}{#1}}
\define@key{prop@gen}{fundsuntil}{\def\prop@gen@fundsuntil{#1}%
\pdata@def{prop}{gen}{fundsuntil}{#1}}
\define@key{prop@gen}{topdownPM}[true]{\def\prop@gen@topdownPM{#1}}
\define@key{prop@gen}{botupPM}[true]{\def\prop@gen@botupPM{#1}}
\define@key{prop@gen}{keywords}{\def\prop@gen@keywords{#1}}
%    \end{macrocode}
% and the default values, these will be used, if the author does not specify something
% better.
%
%    \begin{macrocode}
\newcommand\prop@gen@acro@default{ACRONYM}
\def\prop@gen@acro{\prop@gen@acro@default}
\newcommand\prop@gen@months@default{???months???}
\def\prop@gen@months{\prop@gen@months@default}
\newcommand\prop@gen@title@default{???Proposal Title???}
\def\prop@gen@title{\prop@gen@title@default}
\newcommand\prop@gen@instrument@default{??? Instrument ???}
\def\prop@gen@instrument{\prop@gen@instrument@default}
%    \end{macrocode}
%
% \begin{macro}{\prop@tl}
%   An auxiliary macro that is handy for making tables of WorkAddress data. 
%    \begin{macrocode}
\newcommand\prop@tl[2]{\xdef\tab@line{}
\@for\tl@ext:={#1}\do{\xdef\tab@line{\tab@line&#2}}
\tab@line}
%    \end{macrocode}
% \end{macro}
%
% \subsection{Proposal Appearance}\label{sec:impl:appearance}
% 
% We define the keys for the proposal appearance
%    \begin{macrocode}
\def\prop@gen@compactht{false}
\define@key{prop@gen}{compactht}[true]{\def\prop@gen@compactht{#1}}
%</pdata>
%    \end{macrocode}
%
% \begin{environment}{emphbox}
%    \begin{macrocode}
%<*sty>
\newmdenv[settings=\large]{emphbox}
%    \end{macrocode}
% \end{environment}
% 
% \subsection{The proposal Environment and Title Page}\label{sec:impl:titlepage}
%  
% \begin{environment}{prop@proposal}
%   This internal environment is called in the |proposal| environment from the |proposal|
%   class. The implementation here is only a stub to be substituted in a specialized
%   class.
%    \begin{macrocode}
\newenvironment{prop@proposal}
{\thispagestyle{empty}%
\begin{center}
  {\LARGE \prop@gen@instrument}\\[.2cm]
  {\LARGE\textbf{\prop@gen@title}}\\[.3cm]
  \ifx\prop@gen@acronym\@empty\else{\LARGE Acronym: {\prop@gen@acronym}}\\[.2cm]\fi
  {\large\today}\\[1em]
  \begin{tabular}{c*{\the@PIs}{c}}
    \prop@tl\prop@gen@PIs{\wa@ref3{person}\tl@ext{name}}\\
    \prop@tl\prop@gen@PIs{\wa@ref3{institution}{\wa@ref3{person}\tl@ext{affiliation}}{name}}
 \end{tabular}\\[2cm]
\end{center}
\setcounter{tocdepth}{2}\tableofcontents\newpage\setcounter{page}{1}}
%    \end{macrocode}
% Now we come to the end of the environment:
%    \begin{macrocode}
{\section{List of Attachments}
\begin{itemize}
\@for\@I:=\prop@gen@PIs\do{%
\item Curriculum Vitae and list of publications for
  \wa@ref3{person}\@I{personaltitle} \wa@ref3{person}\@I{name}}
\end{itemize}\newpage
\printbibliography[heading=warnpubs]}
%    \end{macrocode}
% \end{environment}
% 
% \begin{environment}{proposal}
%   The |proposal| environment reads the metadata keys defined above, and if there were no
%   |site| keys, then it defines keys |RM| and |RAM| (unless the |noRAM| package option
%   was given) for the |workpackage| environment. Also it reads the project data file and
%   opens up the project data file |\pdata@out|, which it also closes at the end. 
% 
%   The environment calls an internal version of the environment |prop@proposal| that can
%   be customized by the specializing classes.
%    \begin{macrocode}
\newenvironment{proposal}[1][]{\readpdata\jobname
\ofpagetrue\setkeys{prop@gen}{#1}
\pdata@open\jobname
\if@sites\else
\define@key{workpackage}{RM}{\pdata@def{wp}\wp@id{RM}{##1}\@dmp{RM=##1}}
\if@RAM\define@key{workpackage}{RAM}{\pdata@def{wp}\wp@id{RAM}{##1}\@dmp{RAM=##1}}\fi
\define@key{task}{RM}{\pdata@def{task}{\wp@id @\task@id}{RM}{##1}\@dmp{RM=##1}}
\if@RAM\define@key{task}{RAM}{\pdata@def{wp}{\wp@id @\task@id}{RAM}{##1}\@dmp{RAM=##1}}\fi
\fi
\newcounter{@PIs}
\@ifundefined{prop@gen@PIs}{}{\@for\@I:=\prop@gen@PIs\do{\stepcounter{@PIs}}}
\newcounter{@sites}
\@ifundefined{prop@gen@sites}{}{\@for\@I:=\prop@gen@sites\do{\stepcounter{@sites}}}
\setcounter{page}{0}
\begin{prop@proposal}}
%    \end{macrocode}
% Now we come to the end of the environment, we take care of the last page and print the
% references. 
%    \begin{macrocode}
{\end{prop@proposal}
\pdata@def{prop}{page}{last}{\thepage}\ofpagefalse
\pdata@close}
%</sty>
%    \end{macrocode}
% \end{environment}
%
% The |report| environment is similar, but somewhat simpler
%
% \begin{environment}{report}
%    \begin{macrocode}
%<*reporting>
\newif\if@report\@reportfalse
\newenvironment{report}[1][]%
{\@reporttrue\readpdata\jobname%
\ofpagetrue\setkeys{prop@gen}{#1}%
\pdata@open\jobname%
\@ifundefined{prop@gen@PIs}{}{\newcounter{@PIs}\@for\@I:=\prop@gen@PIs\do{\stepcounter{@PIs}}}%
\@ifundefined{prop@gen@sites}{}{\newcounter{@sites}\@for\@I:=\prop@gen@sites\do{\stepcounter{@sites}}}%
\setcounter{page}{0}%
\begin{prop@report}}
{\end{prop@report}%
\pdata@def{prop}{page}{last}{\thepage}\ofpagefalse\newpage
\printbibliography[heading=warnpubs]
\pdata@close}
%    \end{macrocode}
% \end{environment}
%
% \begin{environment}{prop@report}
%    \begin{macrocode}
\newenvironment{prop@report}
{\begin{center}
  {\LARGE Final Project Report}\\[.2cm]
  {\LARGE\textbf{\prop@gen@title}}\\[.3cm]
  \ifx\prop@gen@acronym\@empty\else{\LARGE Acronym: {\prop@gen@acronym}}\\[.2cm]\fi
  {\large\today}\\[1em]
  \begin{tabular}{c*{\the@PIs}{c}}
    \prop@tl\prop@gen@PIs{\wa@ref3{person}\tl@ext{name}}\\
    \prop@tl\prop@gen@PIs{\wa@ref3{institution}{\wa@ref3{person}\tl@ext{affiliation}}{name}}
 \end{tabular}\\[2cm]
\end{center}
\setcounter{tocdepth}{2}\tableofcontents\newpage\setcounter{page}{1}}
{}
%</reporting>
%    \end{macrocode}
% \end{environment}
% 
% \begin{macro}{\site*}
%    \begin{macrocode}
%<*sty>
\newcommand\site[1]{\hyperlink{site@#1@target}{\wa@ref3{institution}{#1}{acronym}}}
\newcommand\sitename[1]{\hyperlink{site@#1@target}{\wa@ref3{institution}{#1}{name}}}
%    \end{macrocode}
% \end{macro}
%
% \subsection{Objectives}\label{sec:impl:objectives}
%    
% We first define a presentation macro for objectives
% \begin{macro}{\objective@label}
%    \begin{macrocode}
\newcommand\objective@label[1]{O#1}
%    \end{macrocode}
% \end{macro}
% We define the keys for the objectives environment
%    \begin{macrocode}
\define@key{obj}{id}{\def\obj@id{#1}\@dmp{id=#1}}
\define@key{obj}{title}{\def\obj@title{#1}\@dmp{title=#1}}
\define@key{obj}{short}{\def\obj@short{#1}\@dmp{short=#1}}
%    \end{macrocode}
% And a counter for numbering objectives
%    \begin{macrocode}
\newcounter{objective}
%    \end{macrocode}
% \begin{environment}{objective}
%    \begin{macrocode}
\newenvironment{objective}[1][]
{\let\obj@id\relax\let\obj@title\relax\let\obj@short\relax%
\setkeys{obj}{#1}\stepcounter{objective}%
\goodbreak\smallskip\par\noindent%
\textbf{\objective@label{\arabic{objective}}:%
~\pdata@target{obj}{\obj@id}{\pdataref{obj}{\obj@id}{title}}\ignorespaces}%
\pdata@def{obj}\obj@id{label}{\objective@label\theobjective}%
\@ifundefined{obj@title}{}{\pdata@def{obj}\obj@id{title}\obj@title}%
\@ifundefined{obj@short}{}{\pdata@def{obj}\obj@id{short}\obj@short}}
{}
%    \end{macrocode}
% \end{environment}
%
% \begin{macro}{\OBJref}
%    \begin{macrocode}
\newcommand\OBJref[1]{\pdataRef{obj}{#1}{label}}
\newcommand\OBJtref[1]{\OBJref{#1}: \pdataRefFB{obj}{#1}{short}{title}}
%    \end{macrocode}
% \end{macro}
% 
% \subsection{Work Areas and Work Packages}\label{sec:impl:wpwa}
%    
% We first define keys for  work areas (if we are in larger project). 
%    \begin{macrocode}
\ifwork@areas
\define@key{workarea}{id}{\def\wa@id{#1}\@dmp{id=#1}}
\define@key{workarea}{title}{\pdata@def{wa}\wa@id{title}{#1}}
\define@key{workarea}{short}{\pdata@def{wa}\wa@id{short}{#1}}
\define@key{workarea}{lead}{\pdata@def{wa}\wa@id{lead}{#1}}
\fi
%    \end{macrocode}
% work packages have similar ones. 
%    \begin{macrocode}
\define@key{workpackage}{id}{\def\wp@id{#1}\@dmp{id=#1}}
\define@key{workpackage}{title}{\pdata@def{wp}\wp@id{title}{#1}}
\define@key{workpackage}{short}{\pdata@def{wp}\wp@id{short}{#1}}
\define@key{workpackage}{lead}{\pdata@def{wp}\wp@id{lead}{#1}\def\wp@lead{#1}\@dmp{lead=#1}}
\define@key{workpackage}{type}{\def\wp@type{#1}\pdata@def{wp}\wp@id{type}{#1}}
\define@key{workpackage}{wphases}{\def\wp@wphases{#1}\pdata@def{wp}\wp@id{wphases}{#1}}
\define@key{workpackage}{swsites}[true]{\def\wp@swsites{#1}}
%    \end{macrocode}
% We define the constructors for the work package and work area labels and titles. 
%    \begin{macrocode}
\newcommand\wp@mk@title[1]{Work Package {#1}}
\newcommand\wp@label[1]{WP{#1}}
\ifwork@areas
\newcommand\wa@label[1]{WA{#1}}
\newcommand\wa@mk@title[1]{Work Area {#1}}
\fi
%    \end{macrocode}
% The |wa| and |wp| counters are for the work packages and work areas, the counter
% |deliv| for deliverables.
%    \begin{macrocode}
\ifwork@areas\newcounter{wa}\newcounter{wp}[wa]\else\newcounter{wp}\fi
\ifdelivs\newcounter{deliv}[wp]\fi
\newcounter{allwp}
%    \end{macrocode}
% 
% \begin{macro}{\update@*}
%   update the list |\@wps| of the work packages in the local group and the list |\@was|
%   work areas for the staff efforts table: if |\@wps| is undefined, then initialize the
%   comma-separated list, otherwise extend it.\ednote{with the current architecture, we
%   cannot have work areas that do not contain work packages, this leads to the error that
%   |\@wps| is undefined in |end{workplan}|}
%    \begin{macrocode}
\newcommand\update@wps[1]{\@ifundefined{@wps}{\xdef\@wps{#1}}{\xdef\@wps{\@wps,#1}}}
\newcommand\update@tasks[1]{\@ifundefined{@tasks}{\xdef\@tasks{#1}}{\xdef\@tasks{\@tasks,#1}}}
\newcommand\update@deps[1]{\@ifundefined{task@deps}{\xdef\task@deps{#1}}{\xdef\task@deps{\task@deps,#1}}}
\ifwork@areas\def\update@was#1{\@ifundefined{@was}{\xdef\@was{#1}}{\xdef\@was{\@was,#1}}}\fi
%    \end{macrocode}
% \end{macro}
%
% \begin{macro}{\decode@wphase}
%   |\decode@wphase| decodes a string of the form \meta{start}|-|\meta{end}|!|\meta{force}
%   and defines the macros |\wphase@start|, |\wphase@end|, and |\wphase@force| with the
%   three parts and also computes |\wphase@len|. The intermediate parsing macro
%   |\decode@p@start| parses out the start (a number), and passes on to |\decode@p@end|,
%   which parses out the end (another number) and the force string, which is either empty
%   (if the |!|\meta{force} part is omitted) or of the form |!|\meta{force}. In the first
%   case the default value 1 is returned for |\decode@force| in the second \meta{force}.
%    \begin{macrocode}
\newcommand\decode@wphase[1]{\expandafter\decode@p@start#1@%
\local@count\wphase@end\advance\local@count by -\wphase@start%
\def\wphase@len{\the\local@count}}
\def\decode@p@start#1-#2@{\def\wphase@start{#1}\decode@p@end#2!@}
\def\decode@p@end#1!#2@{\def\wphase@end{#1}\def\@test{#2}%
\ifx\@test\@empty\def\wphase@force{1}\else\decode@p@force#2\fi}
\def\decode@p@force#1!{\def\wphase@force{#1}}
%    \end{macrocode}
% \end{macro}
%
% \begin{macro}{\startend@wphases}
%   We first iteratively decode the work phases, so that the last definition of
%   |\wphase@end| remains, then we parse out the start of the first workphase to define
%   |\wphase@start|
%    \begin{macrocode}
\def\wphases@start#1-#2@{\def\wphase@start{#1}}
\newcommand\startend@wphases[1]{\def\@test{#1}
\ifx\@test\@empty\def\wphase@start{0}\def\wphase@end{0}\else%
\@for\@I:=#1\do{\expandafter\decode@p@start\@I @}
\expandafter\wphases@start#1@\fi}
%    \end{macrocode}
% \end{macro}
%
% with these it is now relatively simple to define the interface macros. 
%
% \begin{environment}{work@package}
%   The |workpackage| environment collects the keywords, steps the counters, writes the metadata
%   to the aux file, updates the work packages in the local group, generates the work
%   package number |\wp@num|.
%    \begin{macrocode}
\newcounter{wp@RM}
\if@RAM\newcounter{wp@RAM}\fi
\newenvironment{work@package}[1][]%
{\def\wp@wphases{0-0}% default values
\def\wp@swsites{false}
\setkeys{workpackage}{#1}\stepcounter{wp}\stepcounter{allwp}%
\startend@wphases\wp@wphases%
\pdata@def{wp}\wp@id{start}\wphase@start\pdata@def{wp}\wp@id{end}\wphase@end%
\@ifundefined{wp@type}{}{\pdata@def{wp}\wp@id{type}\wp@type}%
\let\@tasks=\relax%
\edef\wp@num{\ifwork@areas\thewa.\fi\thewp}%
\pdata@def{wp}\wp@id{label}{\wp@label\wp@num}%
\pdata@def{wp}\wp@id{number}{\thewp}%
\pdata@def{wp}\wp@id{page}{\thepage}%
\update@wps\wp@id%
\edef\wp@num{\ifwork@areas\thewa.\fi\thewp}%
\pdata@def{wp}{\wp@id}{num}{\thewp}%
%    \end{macrocode}
% If we have sites, we have to compute the total RM and RAM for this WP.
%    \begin{macrocode}
\if@sites%
\setcounter{wp@RM}{0}\if@RAM\setcounter{wp@RAM}{0}\fi%
\@for\@site:=\prop@gen@sites\do{%
\edef\@RM{\pdataref@num\wp@id\@site{RM}}\addtocounter{wp@RM}{\@RM}%
\if@RAM\edef\@RAM{\pdataref@num\wp@id\@site{RAM}}\addtocounter{wp@RAM}{\@RAM}\fi}
\pdata@def{wp}\wp@id{RM}{\thewp@RM}%
\if@RAM\pdata@def{wp}\wp@id{RAM}{\thewp@RAM}\fi%
\fi}% if@sites
{\@ifundefined{@tasks}{}{\pdata@def{\wp@id}{task}{ids}\@tasks}}
%    \end{macrocode}
%  \end{environment}
%
% \begin{environment}{workpackage}
%   With this, it becomes simple to define a work package environment. We consider two
%   cases, if we have sites, then we make a header table. If not, we can make things much
%   simpler: we just generate a subsection
%    \begin{macrocode}
\newenvironment{workpackage}[1][]%
{\begin{work@package}[#1]%
\ifgrantagreement\else%
%\if@wpsubsection\subsubsection*{{\wp@mk@title\thewp}: \pdataref{wp}\wp@id{title}}\fi
\if@sites\goodbreak\medskip\wpheadertable%
\else\subsubsection*{{\wptitle} (\wprm)}\fi%
\addcontentsline{toc}{paragraph}{{\wp@mk@title\thewp}: \pdataref{wp}\wp@id{title}}%
\noindent\ignorespaces%
\fi}
{\end{work@package}}
%    \end{macrocode}
%  \end{environment}
%
% \begin{macro}{\wptitle}
%   \ednote{document above}
%    \begin{macrocode}
\newcommand\wptitle{\wp@mk@title{\wp@num}: \pdata@target{wp}{\wp@id}{\pdataref{wp}\wp@id{title}}}
%    \end{macrocode}
% \end{macro}
%
% \begin{macro}{\wprm}
% \ednote{document above}
%    \begin{macrocode}
\newcommand\wprm{\pdataref@safe{wp}\wp@id{RM}\if@RAM\ RM+\pdataref{wp}\wp@id{RAM} RAM\fi}
%    \end{macrocode}
% \end{macro}
%
% \begin{macro}{\if@site@contributes}
%   Called as |\if@site@contributes{|\meta{site}|}{|\meta{tokens}|}| the following
%   happens: If |\prop@gen@compactht| is |\@true| (set by the |compactht| attribute on the
%   |proposal| environment), then \meta{tokens} is processed. Otherwise, \meta{tokens} is
%   only processed if \meta{site} contributes to the current work package (i.e the |RM|
%   $\ne0$ and |RAM| $\ne0$)
%    \begin{macrocode}
\newcount\site@contribution%
\newcommand\if@site@contributes[2]{%
\ifx\prop@gen@compactht\@true
\if@RAM\ifnum\pdataref@num\wp@id{#1}{RM} > 0 \ifnum \pdataref@num\wp@id{#1}{RAM} > 0 #2\fi\fi
\else\ifnum\pdataref@num\wp@id{#1}{RM} > 0 #2\fi\fi
\else #2\fi}
%    \end{macrocode}
% \end{macro}
%
% The following macro computes the sites line (in the token register
% \DescribeMacro{\wp@sites@line}|\wp@sites@line|), the efforts line (in
% \DescribeMacro{\wp@efforts@line}|\wp@efforts@line|), and the sites number (in the
% counter \DescribeMacro{\wp@sites@num}|\sites@num|) for later inclusion in the
% |\wpheadertable|. If |\prop@gen@compactht| is |\@true|, then no sites without
% contributions are listed in the table.
%    \begin{macrocode}
\newcounter{wp@sites@num}
\newcommand\wp@sites@efforts@lines{%
\setcounter{wp@sites@num}{0}
{\let\G@refundefinedtrue=\relax\let\@latex@warning=\relax\let\@sw\relax%
\let\site\relax\let\textbf\relax\let\sum@style\relax\let\lead@style\relax%
\let\pn\relax\let\sys\relax%
\xdef\wp@sites@line{\wp@legend@site}\xdef\wp@efforts@line{\wp@legend@effort}%initialize lines
\@for\@site:=\prop@gen@sites\do{\if@site@contributes\@site{\stepcounter{wp@sites@num}}%
\xdef\wp@sites@line{\wp@sites@line%
\if@site@contributes\@site{&%
\ifx\wp@swsites\@true%
\@sw{\ifx\@site\wp@lead\lead@style{\site{\@site}}\else\site{\@site}\fi}%
\else\ifx\@site\wp@lead\lead@style{\site{\@site}}\else\site{\@site}\fi%
\fi}}%
\xdef\wp@efforts@line{\wp@efforts@line%
\if@site@contributes\@site{&%
\ifx\@site\wp@lead%
\lead@style{\pdataref@safe\wp@id\@site{RM}\if@RAM+\pdataref@safe\wp@id\@site{RAM}\fi}
\else\pdataref@safe\wp@id\@site{RM}\if@RAM+\pdataref@safe\wp@id\@site{RAM}\fi\fi}}%
}% do
\xdef\wp@sites@line{\wp@sites@line&\sum@style{\wp@legend@all}}%
\xdef\wp@efforts@line{\wp@efforts@line&
\sum@style{\textbf{\pdataref{wp}\wp@id{RM}\if@RAM+\pdataref{wp}\wp@id{RAM}\fi}}}}}
%    \end{macrocode}
% 
% \begin{macro}{\wpheadertable}
%   This macro computes the default work package header table, if there are sites.
%    \begin{macrocode}
\newcommand\wpheadertable{%
\wp@sites@efforts@lines%
\par\noindent\begin{tabular}{|l||l|*{\thewp@sites@num}{c|}|c|}\hline%
\textbf{\wp@mk@title{\wp@num}}&\wp@sites@line\\\hline%
\textsf{\pdata@target{wp}{\wp@id}{\pdataref{wp}\wp@id{title}}} &\wp@efforts@line\\\hline%
\end{tabular}\smallskip\par\noindent\ignorespaces}
%    \end{macrocode}
% and now multilinguality support
%    \begin{macrocode}
\newcommand\wp@legend@site{Site}
\newcommand\wp@legend@effort{Effort\if@RAM{ (RM+RAM)}\fi}
\newcommand\wp@legend@all{\textbf{all}}
%    \end{macrocode}
% \end{macro}
% 
% \begin{environment}{workarea}
%   the |workarea| environment for work areas is almost the same, but we also have to
%   initialize the work package counters. Also, the efforts can be computed from the work
%   packages in this group via the |wa@effort| counter
%    \begin{macrocode}
\newcounter{prop@RM}\if@RAM\newcounter{prop@RAM}\fi
\ifwork@areas
\newcounter{wa@RM}\if@RAM\newcounter{wa@RAM}\fi\newcounter{wa@wps}
\newenvironment{workarea}[1][]
{\setkeys{workarea}{#1}
\let\@wps=\relax
\stepcounter{wa}
\pdata@def{wa}{\wa@id}{label}{\wa@label\thewa}
\pdata@def{wa}{\wa@id}{number}{\thewa}
\pdata@def{wa}{\wa@id}{page}{\thepage}
\update@was{\wa@id}
\pdata@def{wa}{\wa@id}{num}{\thewa}
\setcounter{wa@RM}{0}\if@RAM\setcounter{wa@RAM}{0}\fi\setcounter{wa@wps}{0}
\edef\@@wps{\pdataref@aux\wa@id{wp}{ids}}
\@for\@wp:=\@@wps\do{\stepcounter{wa@wps}%
\if@sites
\@for\@site:=\prop@gen@sites\do{%
  \edef\@RM{\pdataref@num\@wp\@site{RM}}
  \if@RAM\edef\@RAM{\pdataref@num\@wp\@site{RAM}}\fi
  \addtocounter{wa@RM}{\@RM}\addtocounter{prop@RM}{\@RM}
  \if@RAM\addtocounter{wa@RAM}{\@RAM}\addtocounter{prop@RAM}{\@RAM}\fi}
\else
\edef\@RM{\pdataref@num{wp}\@wp{RM}}
\if@RAM\edef\@RAM{\pdataref@num{wp}\@wp{RAM}}\fi
\addtocounter{wa@RM}{\@RM}\addtocounter{prop@RM}{\@RM}
\if@RAM\addtocounter{wa@RAM}{\@RAM}\addtocounter{prop@RAM}{\@RAM}\fi
\fi}
\pdata@def{wa}\wa@id{RM}\thewa@RM
\pdata@def{prop}{all}{RM}\theprop@RM
\if@RAM
\pdata@def{wa}\wa@id{RAM}\thewa@RAM
\pdata@def{prop}{all}{RAM}\theprop@RAM
\fi
\subsubsection*{{\wa@mk@title\thewa}: {\pdata@target{wa}\wa@id{\pdataref{wa}\wa@id{title}}}}
\addcontentsline{toc}{subsubsection}{{\wa@mk@title\thewa}: \pdataref{wa}\wa@id{title}}%
\ignorespaces}
{\@ifundefined{@wps}{}{\pdata@def\wa@id{wp}{ids}\@wps}\pdata@def\wa@id{wp}{count}\thewa@wps}\fi
%    \end{macrocode}
% \end{environment}
% 
% \begin{environment}{workplan}
%   The |workplan| environment sets up the accumulator macros |\@wps|, |\@was|, for the
%   collecting the identifiers of work packages and work areas. At the end of the
%   workplan description it writes out their content to the aux file for reference.
%    \begin{macrocode}
\ifdelivs\newwrite\wpg@delivs\fi
\newenvironment{workplan}%
{\ifdelivs\immediate\openout\wpg@delivs=\jobname.delivs\fi
\ifwork@areas\let\@was=\relax\else\let\@wps=\relax\fi}%
{\@ifundefined{task@deps}{}{\pdata@def{all}{task}{deps}{\task@deps}}
\pdata@def{all}{task}{count}{\thealltasks}
\ifwork@areas
\@ifundefined{@was}{}{\pdata@def{all}{wa}{ids}\@was}
\else
\@ifundefined{@wps}{}{\pdata@def{all}{wp}{ids}\@wps}
\fi
\ifdelivs\@ifundefined{mile@stones}{}
{\@for\@I:=\mile@stones\do{%
\pdata@def{mile}\@I{delivs}{\@ifundefined{\@I delivs}{}{\csname\@I delivs\endcsname}}}}\fi
\ifwork@areas\pdata@def{all}{wa}{count}{\thewa}\fi
\pdata@def{all}{wp}{count}{\theallwp}
\ifdelivs
\pdata@def{all}{deliverables}{count}{\thedeliverable}
\pdata@def{all}{milestones}{count}{\themilestone}
\fi
\ifdelivs\closeout\wpg@delivs\fi}
%    \end{macrocode}
% \end{environment}
%
% \subsection{Tasks}\label{sec:impl:tasks}
% 
% \begin{environment}{tasklist}
%    \begin{macrocode}
\newenvironment{tasklist}
{\smallskip\begin{compactenum}}{\end{compactenum}\smallskip}
%    \end{macrocode}
% \end{environment}
% The next step is to 
%    \begin{macrocode}
\ifwork@areas
\newcommand\task@label[3]{\textbf{T#1.#2.#3}}
\else
\newcommand\task@label[2]{\textbf{T#1.#2}}
\fi
%    \end{macrocode}
% We define the keys for the task macro
%    \begin{macrocode}
\define@key{task}{id}{\def\task@id{#1}\@dmp{id=#1}}
\define@key{task}{wphases}{\def\task@wphases{#1}\@dmp{wphases=#1}}
\define@key{task}{requires}{\@requires\task@id{#1}\@dmp{req=#1}}
\define@key{task}{title}{\def\task@title{#1}}
\define@key{task}{lead}{\def\task@lead{#1}}
\define@key{task}{partners}{\def\task@partners{#1}}
\define@key{task}{PM}{\def\task@PM{#1}}
\define@key{task}{issue}{\def\task@issue{#1}}
\def\task@set#1{\edef\task@id{task\thetask@all}
\def\task@wphases{0-0}\def\task@partners{}\def\task@lead{}\def\task@PM{}\def\task@title{}
\setkeys{task}{#1}}
%    \end{macrocode}
% 
% \begin{macro}{\task@post@title@space}
%   make the space after the title tweakable
%    \begin{macrocode}
\def\task@post@title@space{\;}
%    \end{macrocode}
% \end{macro}
% 
% \begin{environment}{task}
%   The |task| environment. We first set up config stuff 
%    \begin{macrocode}
\newcounter{alltasks}
\def\task@post@title@space{ }
\newcommand\task@legend@partners{Sites: }
\newcommand\task@legend@PM{PM}
%    \end{macrocode}
% now comes the environment proper. We first call |\@task| on the keyval argument to do
% the metadata handling. Then we start formatting the task as an item in the |description|
% list from the |tasklist| environment, and print the |title| if there is one 
%    \begin{macrocode}
\newenvironment{task}[1][]%
{\stepcounter{alltasks}%
\@task{#1}%
\item[\pdata@target{task}{\taskin\task@id\wp@id}%
{\ifwork@areas\task@label\thewa\thewp\thetask@wp\else\task@label\thewp\thetask@wp\fi}]%
\ifx\task@title\@empty\else\textbf\task@title\fi\task@post@title@space%
%    \end{macrocode}
% now we decode and show the work phases on the task, if they have been specified.
%    \begin{macrocode}
\def\@initial{0-0}%
\ifx\task@wphases\@initial\else%
\let\@@sep=\relax\@for\@I:=\task@wphases%
\do{\decode@wphase\@I%
\@@sep\show@wphase\wphase@start\wphase@end\wphase@force%
\let\@@sep=\sep@wphases}%
\fi% initial
%    \end{macrocode}
% in non-|submit| mode we give the specified PM for cross-checking
%    \begin{macrocode}
\ifsubmit\else\ifx\task@PM\@empty\else\task@PM~\task@legend@PM;\fi\fi%
%    \end{macrocode}
% and we list the partners who contribute if they are specified. 
%    \begin{macrocode}
\if@sites%
\ifx\task@lead\@empty\else\ \task@legend@partners\site\task@lead~(\legend@lead)%
\@for\@I:=\task@partners\do{, \site\@I}\\\fi%
%    \end{macrocode}
% if there are no partners, then we show the RM/RAM contributions specified (if any)
%    \begin{macrocode}
\ifx\task@partners\@empty
\xdef\@@involvement{}\xdef\@@inv{}%
\xdef\@@sep{, }\def\m@sep{}% do not show the sep the first time around
\edef\@@sites{\prop@gen@sites}%
{\let\site\relax% to to render it inert here
\@for\@site:=\@@sites\do{%
\edef\@@RM{\pdataref@safe{\wp@id @\task@id}\@site{RM}}%
\ifx\@@RM\@empty\else\xdef\@@inv{showit}%
\xdef\@@involvement{\@@involvement% and
\m@sep\site{\@site}: \@@RM\if@RAM\ifx\@@RAM\@empty\else/\@@RM\fi\fi}
\let\m@sep=\@@sep% but the second time show it.
\fi}}% \@@RM empty
\ifx\@@inv\@empty\else(RM{\if@RAM/RAM\fi} distribution: \@@involvement)\strut\\\fi
\fi% no partners key
\fi% sites
%    \end{macrocode}
% finally, we ignore any spaces that may follow the the |task| environment
%    \begin{macrocode}
\ignorespaces}
{\smallskip}
%    \end{macrocode}
% now the multilingual support and presentation configuration
%    \begin{macrocode}
\newcommand\month@label[1]{M#1}
\newcommand\show@wphase[3]{\edef\@test{#3}\def\@one{1}%
\month@label{#1}-\month@label{#2}%
\ifx\@test\@empty\else\ifx\@test\@one\else @#3\fi\fi}
\newcommand\sep@wphases{; }
\newcommand\legend@partners{Partners}
\newcommand\legend@lead{lead}
\newcommand\task@label@long{Task}
%    \end{macrocode}
% \end{environment}
%
% \begin{macro}{\@task}
%   The |\@task| macro is a internal macro which takes a bunch of keyword keys and writes
%   their values to the aux file.
%    \begin{macrocode}
\newcounter{task@all}\newcounter{task@wp}[wp]
\newcount\task@@end
\def\@task#1{\stepcounter{task@all}\stepcounter{task@wp}%
\task@set{#1}%
\pdata@def{task}{\taskin\task@id\wp@id}{title}{\task@title}
\pdata@def{task}{\taskin\task@id\wp@id}{lead}{\task@lead}
\pdata@def{task}{\taskin\task@id\wp@id}{partners}{\task@partners}
\pdata@def{task}{\taskin\task@id\wp@id}{PM}{\task@PM}
\pdata@def{task}{\taskin\task@id\wp@id}{wphases}{\task@wphases}
\@ifundefined{task@issue}{}
{\pdata@def{task}{\taskin\task@id\wp@id}{issue}{\task@issue}}%
\ifwork@areas
\pdata@def{task}{\taskin\task@id\wp@id}{label}{\task@label\thewa\thewp\thetask@wp}%
\else
\pdata@def{task}{\taskin\task@id\wp@id}{label}{\task@label\thewp\thetask@wp}%
\fi
\pdata@def{task}{\taskin\task@id\wp@id}{number}{\thetask@wp}%
\pdata@def{task}{\taskin\task@id\wp@id}{page}{\thepage}%
\update@tasks{\taskin\task@id\wp@id}}
%    \end{macrocode}
% \end{macro}
% 
% \subsection{Work Phase Metadata}\label{sec:impl:workphase}
% 
% \begin{macro}{\workphase}
%    \begin{macrocode}
\newcommand\workphase[1]{\PackageError{proposal}
  {The \protect\workphase macro is deprecated,\MessageBreak
    use the attributes wphase on the workpackage environment instead!}}
%    \end{macrocode}
% \end{macro}
%
% \begin{macro}{\*task*ref}
%    \begin{macrocode}
\newcommand\taskin[2]{#2@#1}
\newcommand\taskref[2]{\pdataRef{task}{#1@#2}{label}}
\newcommand\taskreflong[2]{\pdataRef{task}{#2}{label}}
\newcommand\tasktref[2]{\taskref{#1}{#2}: \pdataRefFB{task}{#1@#2}{short}{title}}
\newcommand\localtaskref[1]{\taskref{\wp@id}{#1}}
\newcommand\localtasktref[1]{\tasktref{\wp@id}{#1}}
%    \end{macrocode}
% \end{macro}
% now we initialize experimental infrastructure for task dependencies (not very well used/tested)
%    \begin{macrocode}
\newcounter{gantt@deps}
\def\@requires#1#2{\stepcounter{gantt@deps}%
\edef\dep@id{taskdep\thegantt@deps}%
\pdata@def{taskdep}\dep@id{from}{\taskin{#1}\wp@id}%
\pdata@def{taskdep}\dep@id{to}{#2}%
\update@deps\dep@id}
%    \end{macrocode}
%
%\subsection{Milestones and Deliverables}\label{sec:impl:deliverables}
%
% \begin{macro}{deliv@error}
%   this macro raises an error if deliverable commands are used without the |deliverables|
%   option being set.
%    \begin{macrocode}
\newcommand\deliv@error{\PackageError{proposal}
{To use use deliverables, you have to specify the option 'deliverables'}}
%    \end{macrocode}
% \end{macro}
%
% \begin{environment}{wpdelivs}
%    \begin{macrocode}
\newenvironment{wpdelivs}{\begin{wp@delivs}}{\end{wp@delivs}}
%    \end{macrocode}
% \end{environment}
% 
% \begin{environment}{wp@delivs}
%    \begin{macrocode}
\newenvironment{wp@delivs}
{\ifdelivs\textbf\deliv@legend@delivs:\\[-3ex]%
\begin{compactdesc}\else\deliv@error\fi}
{\ifdelivs\end{compactdesc}\fi}
%    \end{macrocode}
% and now multilinguality support 
%    \begin{macrocode}
\newcommand\deliv@legend@delivs{Deliverables}
%    \end{macrocode}
% \end{environment}
% 
% \begin{macro}{\wadelivs}
%    \begin{macrocode}
\newenvironment{wadelivs}
{\textbf\deliv@legend@delivs:\\[-3ex]\begin{wp@delivs}}
{\end{wp@delivs}}
%    \end{macrocode}
% \end{macro}
%
% \begin{macro}{\lec}
%   This macro is generally useful to put a comment at the end of the line, possibly
%   making a new one if there is not enough space.
%    \begin{macrocode}
\newcommand\lec[1]{\strut\hfil\strut\null\nobreak\hfill\hbox{$\leadsto$#1}\par}
%    \end{macrocode}
% \end{macro}
% 
% \begin{macro}{\deliv@label}
%    \begin{macrocode}
\newcommand\deliv@label[1]{D{#1}}
%    \end{macrocode}
% \end{macro}
% 
% \begin{macro}{\*deliv*ref}
%   This macro is generally useful to put a comment at the end of the line, possibly
%   making a new one if there is not enough space.
%    \begin{macrocode}
\newcommand\delivref[2]{\pdataRef{deliv}{#1@#2}{label}}
\newcommand\localdelivref[1]{\delivref{\wp@id}{#1}}
\newcommand\delivtref[2]{\delivref{#1}{#2}: \pdataRefFB{deliv}{#1@#2}{short}{title}}
\newcommand\localdelivtref[1]{\delivtref{\wp@id}{#1}}
%    \end{macrocode}
% \end{macro}
% 
% \begin{macro}{\wpg@deliv}
%    We first define the keys
%    \begin{macrocode}
\define@key{deliv}{id}{\def\deliv@id{#1}}
\define@key{deliv}{due}{\def\deliv@due{#1}}
\define@key{deliv}{dissem}{\def\deliv@dissem{#1}}
\define@key{deliv}{nature}{\def\deliv@nature{#1}}
\define@key{deliv}{miles}{\def\deliv@miles{#1}}
\define@key{deliv}{short}{\def\deliv@short{#1}}
\define@key{deliv}{lead}{\def\deliv@lead{#1}}
\define@key{deliv}{issue}{\def\deliv@issue{#1}}
\define@key{deliv}{status}{\def\deliv@status{#1}}
\define@key{deliv}{blog}{\def\deliv@blog{#1}}
%    \end{macrocode}
% The |\wpdeliv| macro cycles over the due dates and generates the relevant entries into
% the deliverables file. The first step is to write the general metadata to the pdata
% file.
%    \begin{macrocode}
\newcounter{deliverable}
\newcommand{\wpg@deliv}[3]{% keys, title, type
\stepcounter{deliverable}
\let\deliv@miles=\relax% clean state
\def\@type{#3}\def\@wp{wp}% set up ifx
\def\wpg@id{\csname #3@id\endcsname}
\setkeys{deliv}{#1}\stepcounter{deliv}% set state
\ifx\@type\@wp\def\current@label{\deliv@label{\ifwork@areas\thewa.\fi\thewp.\thedeliv}}
\else\def\current@label{\deliv@label{\thewa.\thedeliv}}\fi
\pdata@def{deliv}{\taskin\deliv@id\wpg@id}{label}{\current@label}
\pdata@def{deliv}{\taskin\deliv@id\wpg@id}{title}{#2}
\pdata@def{deliv}{\taskin\deliv@id\wpg@id}{page}{\thepage}%
\@ifundefined{deliv@short}
{\pdata@def{deliv}{\taskin\deliv@id\wpg@id}{short}{#2}}
{\pdata@def{deliv}{\taskin\deliv@id\wpg@id}{short}{\deliv@short}}
\@ifundefined{deliv@nature}
{\protect\G@refundefinedtrue\@latex@warning{key 'nature' for Deliv \wpg@id undefined}}
{\pdata@def{deliv}{\taskin\deliv@id\wpg@id}{nature}{\deliv@nature}}
\@ifundefined{deliv@dissem}
{\protect\G@refundefinedtrue\@latex@warning{key 'dissem' for Deliv \wpg@id undefined}}
{\pdata@def{deliv}{\taskin\deliv@id\wpg@id}{dissem}{\deliv@dissem}}
\@ifundefined{deliv@lead}
{\protect\G@refundefinedtrue\@latex@warning{key 'lead' for Deliv \wpg@id  undefined}}
{\pdata@def{deliv}{\taskin\deliv@id\wpg@id}{lead}{\deliv@lead}}
\@ifundefined{deliv@due}{}{\pdata@def{deliv}{\taskin\deliv@id\wpg@id}{due}{\deliv@due}}
\@ifundefined{deliv@issue}{}{\pdata@def{deliv}{\taskin\deliv@id\wpg@id}{issue}{\deliv@issue}}
\@ifundefined{deliv@status}{}{\pdata@def{deliv}{\taskin\deliv@id\wpg@id}{status}{\deliv@status}}
\@ifundefined{deliv@blog}{}{\pdata@def{deliv}{\taskin\deliv@id\wpg@id}{blog}{\deliv@blog}}
%    \end{macrocode}
% Then we iterate over the due dates and generate an entry for teach of them.
%    \begin{macrocode}
\@ifundefined{deliv@due}{}{%
\@for\@I:=\deliv@due\do{\protected@write\wpg@delivs{}{\string\deliverable%
{\ifnum\@I<10 0\@I\else\@I\fi}% sort key
{\@I}% due date
{\current@label}% label
{\@ifundefined{deliv@id}{??}{\taskin\deliv@id\wpg@id}}% id
{\@ifundefined{deliv@dissem}{??}{\deliv@dissem}}% dissemination level
{\@ifundefined{deliv@nature}{??}{\deliv@nature}}% nature
{#2}
{\ifx\@type\@wp{WP\ifwork@areas\thewa.\fi\thewp}\else{WA\thewa}\fi}%WP
{\@ifundefined{deliv@lead}{??}{\string\site{\deliv@lead}}}}}}% lead
%    \end{macrocode}
% And finally, we generate the entry into the deliverables table. 
%    \begin{macrocode}
\item[\current@label\ (%
\delivs@legend@due: \@ifundefined{deliv@due}{??}{\deliv@due},
\delivs@legend@nature: \@ifundefined{deliv@nature}{??}{\deliv@nature},
\delivs@legend@dissem: \@ifundefined{deliv@dissem}{??}{\deliv@dissem},
\delivs@legend@lead: \@ifundefined{deliv@lead}{??}{\site{\deliv@lead}})]
 \pdata@target{deliv}{\taskin\deliv@id\wpg@id}{\textit{#2}}
\@ifundefined{deliv@miles}{}{% print the milestones and update their deliverables
\let\m@sep=\relax% do not print the separator the first time round
\lec{\@for\@I:=\deliv@miles\do{% Iterate over the milestones mentioned
\m@sep\pdataRef{mile}{\@I}{label}% print the milestone reference
\let\m@sep=,}}%set the separator for the next times
\def\d@sep{,}
 \@for\@I:=\deliv@miles\do{% Iterate over the milestones mentioned
 \expandafter\ifx\csname\@I delivs\endcsname\relax% Check that the miles@delivs is empty
  {\expandafter\xdef\csname\@I delivs\endcsname{\wpg@id @\deliv@id}}% if so, skip the separator
   \else\expandafter\xdef\csname\@I delivs\endcsname%if not add it
      {\csname\@I delivs\endcsname\d@sep\wpg@id @\deliv@id}\fi}}}
%    \end{macrocode}
% \end{macro}
%
% Now, we only need to instantiate
% 
% \begin{environment}{wadeliv}
%    \begin{macrocode}
\newenvironment{wadeliv}[2][]{\ifdelivs\wpg@deliv{#1}{#2}{wa}\else\deliv@error\fi}{}
%    \end{macrocode}
% \end{environment}
%
% \begin{environment}{wpdeliv}
%    \begin{macrocode}
\newenvironment{wpdeliv}[2][]{\ifdelivs\wpg@deliv{#1}{#2}{wp}\else\deliv@error\fi}{}
%    \end{macrocode}
% \end{environment}
% 
% \begin{macro}{\milestone@label}
%    \begin{macrocode}
\newcommand\milestone@label[1]{M{#1}}
%    \end{macrocode}
% \end{macro}
% 
% \begin{macro}{\mileref}
%   This macro is generally useful to put a comment at the end of the line, possibly
%   making a new one if there is not enough space.
%    \begin{macrocode}
\newcommand\mileref[1]{\pdataRef{mile}{#1}{label}}
\newcommand\miletref[1]{\mileref{#1}: \pdataRefFB{mile}{#1}{short}{title}}
%    \end{macrocode}
% \end{macro}
% 
% \begin{macro}{\milestone}
%    create a new milestone, initialize its deliverables accumulator macro, set up
%    hyperlinking, and extend the milestones list. 
%    \begin{macrocode}
\newcounter{milestone}
\define@key{milestone}{id}{\gdef\mile@id{#1}}
\define@key{milestone}{month}{\gdef\mile@month{#1}}
\newcommand\milestone[3][]{%
\ifdelivs%
\setkeys{milestone}{#1}\stepcounter{milestone}%
\pdata@def{mile}\mile@id{label}{\milestone@label{\themilestone}}%
\pdata@def{mile}\mile@id{month}{\mile@month}%
\pdata@def{mile}\mile@id{title}{#2}%
\pdata@def{mile}\mile@id{description}{#3}%
\@ifundefined{mile@stones}%
{\xdef\mile@stones{\mile@id}}%
{\xdef\mile@stones{\mile@stones,\mile@id}}%
\@milestone{\mile@id}{#2}{#3}% presentation
\else\deliv@error\fi}
%    \end{macrocode}
% \end{macro}
% 
% \begin{macro}{\@milestone}
%   the corresponding presentation macro.
%    \begin{macrocode}
\newcommand\@milestone[3]{% id, title, description
\item \textbf{\miles@legend@milestone\xspace\pdata@target{mile}\mile@id{\pdataref{mile}{#1}{label}}
(\miles@legend@month \pdataref{mile}\mile@id{month})
\textbf{#2}} #3}
\newcommand\miles@legend@month{Month}
\newcommand\miles@legend@milestone{Milestone}
%    \end{macrocode}
% \end{macro}
% 
% \begin{environment}{milestones}
%   This does the metadata bookkeeping, the layout is delegated to the presentation
%   environment |@milestones| and the legend macros that can be customized for specific
%   proposals.
%    \begin{macrocode}
\newenvironment{milestones}%
{\ifdelivs\begin{@milestones}\else\deliv@error\fi}
{\ifdelivs\pdata@def{all}{mile}{ids}{\mile@stones}%
\pdata@def{all}{mile}{count}{\themilestone}%
\end{@milestones}\fi}
%    \end{macrocode}
% \end{environment}
%
% \begin{environment}{@milestones}
% here we do the work. 
%    \begin{macrocode}
\newenvironment{@milestones}{\begin{enumerate}}{\end{enumerate}}
%    \end{macrocode}
% \end{environment}
%
% \begin{macro}{\deliverable}
%    the first argument is an extended due date to facilitate sorting. 
%    \begin{macrocode}
\newcommand{\deliverable}[9]{\pdataRef{deliv}{#4}{label}&#7&#8&#9&#6&#5&#2\\\hline}%sortkey,due,label,id,title,type,level,lead
%    \end{macrocode}
% \end{macro}
% 
% \begin{environment}{deliverables}
%    
%    \begin{macrocode}
\newenvironment{deliverables}[1]{\ifdelivs\begin{longtable}{|l|p{#1}|l|l|l|l|l|}\hline%
\#&\textbf{\delivs@legend@name}&%
\textbf{\delivs@legend@wp}&%
\textbf{\delivs@legend@lead}&%
\textbf{\delivs@legend@nature}&%
\textbf{\delivs@legend@level}&%
\textbf{\delivs@legend@due}\\\hline\hline%
\endhead%
\else\deliv@error\fi}
{\ifdelivs\end{longtable}\fi}
%    \end{macrocode}
% now the multilingual support
%    \begin{macrocode}
\newcommand\delivs@legend@name{Deliverable name}
\newcommand\delivs@legend@wp{WP}
\newcommand\delivs@legend@nature{Type}
\newcommand\delivs@legend@level{Level}
\newcommand\delivs@legend@due{Due}
\newcommand\delivs@legend@dissem{Dissem.}
\newcommand\delivs@legend@lead{Lead}
%    \end{macrocode}
% \end{environment}
% 
% \begin{macro}{\inputdelivs}
%    
%    \begin{macrocode}
\newcommand{\inputdelivs}[1]{%
\begin{deliverables}{#1}%
\IfFileExists{\jobname.deliverables}%
{\input{\jobname.deliverables}}%
{\IfFileExists{\jobname.delivs}{\input{\jobname.delivs}}{}}
\end{deliverables}}
%</sty>
%    \end{macrocode}
% \end{macro}
% 
% \subsection{Project Data,  Referencing \& Hyperlinking}\label{sec:impl:hyperlinking} 
% 
% \begin{macro}{\pdata@*}
%   |\pdata@out| is the file handle for the project data file, we define internal macros
%   to open and close it.  
%    \begin{macrocode}
%<*pdata>
\newif\ifwork@areas\work@areastrue
\DeclareOption{noworkareas}{\work@areasfalse}
\ProcessOptions
\RequirePackage{xspace}
\newwrite\pdata@out
\newcommand\pdata@open[1]{\immediate\openout\pdata@out=#1.pdata}
\newcommand\pdata@close{\closeout\pdata@out}
%    \end{macrocode}
% \end{macro}
% 
% \begin{macro}{\readpdata}
%   This macro reads the project data file and its error handling
%    \begin{macrocode}
\newcommand\readpdata[1]{\IfFileExists{#1.pdata}
{\message{proposal: Reading Project Data}\makeatletter\input{#1.pdata}\makeatother}
{proposal: No Project Data found, (forward) references may be compromized}}
%    \end{macrocode}
% \end{macro}
%  
% \begin{macro}{\pdata@target}
%   This internal macro makes a hyper-target:
%   |\pdata@target{|\meta{cat}|}{|\meta{id}|}{|\meta{label}|}| prints \meta{label} with a
%   target name \meta{cat}|@|\meta{id}|@target| attached to it.
%    \begin{macrocode}
\newcommand\pdata@target[3]{\hypertarget{#1@#2@target}{#3}}
%    \end{macrocode}
% \end{macro}
%
% \begin{macro}{\pdata@def}
%    This macro writes an |\@pdata@def| command to the current aux file and also executes it. 
%    \begin{macrocode}
\newcommand\pdata@def[4]{%\@pdata@def{#1}{#2}{#3}{#4}%
  \protected@write\pdata@out{}{\string\@pdata@def{#1}{#2}{#3}{#4}}}
%    \end{macrocode}
% \end{macro}
%
% \begin{macro}{\@pdata@def}
%   This macro stores the value of its last argument in a custom macro for reference.
%    \begin{macrocode}
\newcommand\@pdata@def[4]{\expandafter\gdef\csname #1@#2@#3\endcsname{#4}}
%    \end{macrocode}
% \end{macro}
%
% \begin{macro}{\pdataref}
%    \begin{macrocode}
\newcommand\pdataref[3]{\@ifundefined{#1@#2@#3}%
               {\protect\G@refundefinedtrue\@latex@warning{#3 for #1 #2 undefined}??}%
                {\csname #1@#2@#3\endcsname}}%
\newcommand\pdataref@aux[3]{\@ifundefined{#1@#2@#3}{??}{\csname #1@#2@#3\endcsname}}%
\newcommand\pdataref@num[3]{\@ifundefined{#1@#2@#3}{0}{\csname #1@#2@#3\endcsname}}%
\newcommand\pdataref@safe[3]{\@ifundefined{#1@#2@#3}{}{\csname #1@#2@#3\endcsname}}%
%    \end{macrocode}
% \end{macro}
%
% \begin{macro}{\pdatarefFB}
% a variant with fallback field,
%    \begin{macrocode}
\newcommand\pdatarefFB[4]{\@ifundefined{#1@#2@#3}%
{\@ifundefined{#1@#2@#4}%
{\protect\G@refundefinedtrue\@latex@warning{both #3 and its fallback #4 undefined for #1 #2}??}%
{\csname #1@#2@#4\endcsname}}
{\csname #1@#2@#3\endcsname}}
%    \end{macrocode}
% \end{macro}
% 
% \begin{macro}{\pdataRef}
%    \begin{macrocode}
\newcommand\pdataRef[3]{\@ifundefined{#1@#2@#3}%
{\protect\G@refundefinedtrue\@latex@warning{#3 for #1 #2 undefined}??}%
{\hyperlink{#1@#2@target}{\csname #1@#2@#3\endcsname}}}
%    \end{macrocode}
% \end{macro}
% 
% \begin{macro}{\pdataRefFB}
% a variant with fallback field,
%    \begin{macrocode}
\newcommand\pdataRefFB[4]{\@ifundefined{#1@#2@#3}%
{\@ifundefined{#1@#2@#4}%
{\protect\G@refundefinedtrue\@latex@warning{both #3 and its fallback #4 undefined for #1 #2}??}%
{\hyperlink{#1@#2@target}{\csname #1@#2@#4\endcsname}}}
{\hyperlink{#1@#2@target}{\csname #1@#2@#3\endcsname}}}
%    \end{macrocode}
% \end{macro}
% 
% \begin{macro}{\pdatacount}
%    \begin{macrocode}
\newcommand\prop@count[1]{\ifcase #1 zero\or one\or two\or three\or four\or five\or six\or seven \or
  eight\or nine\or ten\or eleven \or twelve\else#1\fi}
\newcommand\pdatacount[2]{\prop@count{\pdataref@num{#1}{#2}{count}}}
%    \end{macrocode}
% \end{macro}
%
% \begin{macro}{\pn*}
%    \begin{macrocode}
\newcommand\pn{\pdataref{prop}{gen}{acronym}\xspace}
\newcommand\pnlong{\pdataref{prop}{gen}{acrolong}\xspace}
%    \end{macrocode}
% \end{macro}
%
% \begin{macro}{\W*ref}
%    \begin{macrocode}
\newcommand\WPref[1]{\pdataRef{wp}{#1}{label}}
\newcommand\WPtref[1]{\WPref{#1}: \pdataRefFB{wp}{#1}{short}{title}}
\ifwork@areas
\newcommand\WAref[1]{\pdataRef{wa}{#1}{label}}
\newcommand\WAtref[1]{\WAref{#1}: \pdataRefFB{wa}{#1}{short}{title}}
\fi
%</pdata>
%    \end{macrocode}
% \end{macro}
%
% \subsection{The Work Package Table}\label{sec:impl:wptable}
%
% \begin{macro}{\*@style}
%   These macros\ednote{maybe add ``wpfig'' in the name to show dependency} determine the
%   styling of cells in the work package table. That can be tweaked by redefining them.
%    \begin{macrocode}
%<*sty>
\definecolorset{gray/rgb/hsb/cmyk}{}{}%
{leadgray,.90/.90,.90,.90/0,0,.90/0,0,0,.10;%
wagray,.70/.70,.70,.70/0,0,.70/0,0,0,.30;%
ganttgray,.60/.60,.60,.60/0,0,.60/0,0,0,.40}
\newcommand\sum@style[1]{\cellcolor{wagray}{\textbf{#1}}}
\newcommand\wa@style[1]{\cellcolor{wagray}{\textbf{#1}}}
\newcommand\wp@style[1]{#1}
\newcommand\lead@style[1]{\cellcolor{leadgray}{\textit{#1}}}
\newcommand\wp@lead@style@explained{light gray italicised}
%    \end{macrocode}
% \end{macro}
% 
% \begin{macro}{\wpfigstyle}
%    \begin{macrocode}
\def\wpfig@style{}
\newcommand\wpfigstyle[1]{\def\wpfig@style{#1}}
%    \end{macrocode}
% \end{macro}
%
%  We first define the options for the |\wpfig| macro, they specify what columns we have
%  in the table. 
%    \begin{macrocode}
\newcounter{wpfig@options}
\define@key{wpfig}{size}{\def\wpfig@size{#1}\@dmp{size=#1}}
\def\@true{true}
\def\wpfig@pages{false}
\define@key{wpfig}{pages}[true]{\def\wpfig@pages{#1}\stepcounter{wpfig@options}}
\def\wpfig@type{false}
\define@key{wpfig}{type}[true]{\def\wpfig@type{#1}\stepcounter{wpfig@options}}
\def\wpfig@start{false}
\define@key{wpfig}{start}[true]{\def\wpfig@start{#1}\stepcounter{wpfig@options}}
\def\wpfig@length{false}
\define@key{wpfig}{length}[true]{\def\wpfig@length{#1}\stepcounter{wpfig@options}}
\def\wpfig@end{false}
\define@key{wpfig}{end}[true]{\def\wpfig@end{#1}\stepcounter{wpfig@options}}
\define@key{wpfig}{label}{\def\wpfig@label{#1}}
\define@key{wpfig}{caption}{\def\wpfig@caption{#1}}
%    \end{macrocode}
%
% \begin{environment}{wp@figure}
%   This environment makes legend for the table (but not the contents) for the |\wpfig|
%   macro. The main work achieved here is to generate the head line (sideways) and the
%   footer in the various cases given by the package options.\ednote{this is a bit of
%   misnomer, it does not do the figure bit. } Depending on the various class and |wpfig|
%   options, we make header and footer line for the table.
%    \begin{macrocode}
\def\@sw#1{\begin{sideways}#1\end{sideways}}
\newenvironment{wp@figure}{\begin{table}[ht]\wpfig@style\begin{center}
{\let\@sw\relax\let\textbf\relax\let\site\relax\let\pn\relax\let\sys\relax%
\gdef\wpfig@headline{\wpfig@legend@wap&\wpfig@legend@title%
\ifx\wpfig@type\@true&\wpfig@legend@type\fi%
\ifx\wpfig@pages\@true&\@sw{\wpfig@legend@page}\fi%
\ifx\wpfig@start\@true&\@sw{\wpfig@legend@start}\fi%
\ifx\wpfig@length\@true&\@sw{\wpfig@legend@length}\fi
\ifx\wpfig@end\@true&\@sw{\wpfig@legend@end}\fi}%
\if@sites%
\@for\@site:=\prop@gen@sites\do{%
\xdef\wpfig@headline{\wpfig@headline&\@sw{\wpfig@legend@siteRM{\@site}}}%
\if@RAM\xdef\wpfig@headline{\wpfig@headline&\@sw{\wpfig@legend@siteRAM{\@site}}}\fi}%
\xdef\wpfig@headline{\wpfig@headline&\@sw{\wpfig@legend@totalRM}}%
\if@RAM\xdef\wpfig@headline{\wpfig@headline&\@sw{\wpfig@legend@totalRAM}}\fi%
\else% if@sites
\xdef\wpfig@headline{\wpfig@headline &\@sw{\wpfig@legend@RM}\if@RAM&\@sw{\wpfig@legend@RAM}\fi}
\fi}%if@sites
\if@RAM\begin{tabular}{|l|l|*{\thewpfig@options}{r|}*{\the@sites}{r|r|}|r|r|}\hline
\else\begin{tabular}{|l|l|*{\thewpfig@options}{r|}|*{\the@sites}{r|}|r|}\hline\fi%|
\wpfig@headline\\\hline\hline}
{\end{tabular}\smallskip\\
\wpfig@legend@RAM@expl\if@sites; \wpfig@legend@lead@expl\fi
\@ifundefined{wpfig@label}{\caption{\wpfig@legend@caption}}{\caption{\wpfig@caption}}
\@ifundefined{wpfig@label}{\label{fig:wplist}}{\label{\wpfig@label}}
\end{center}\end{table}}
%    \end{macrocode}
% and now multilinguality support 
%    \begin{macrocode}
\newcommand\wpfig@legend@wap{\textbf{\ifwork@areas{WA/P}\else{WP}\fi}}
\newcommand\wpfig@legend@title{\textbf{Title}}
\newcommand\wpfig@legend@type{\textbf{type}}
\newcommand\wpfig@legend@page{\textbf{page}}
\newcommand\wpfig@legend@start{\textbf{start}}
\newcommand\wpfig@legend@length{\textbf{length}}
\newcommand\wpfig@legend@end{\textbf{end}}
\newcommand\wpfig@legend@siteRM[1]{\site{#1}\if@RAM\ RM\fi}
\newcommand\wpfig@legend@siteRAM[1]{\site{#1}\ RAM}
\newcommand\wpfig@legend@totalRM{total\if@RAM\ RM\fi}
\newcommand\wpfig@legend@totalRAM{total RAM}
\newcommand\wpfig@legend@RM{RM}
\newcommand\wpfig@legend@RAM{RAM}
\newcommand\wpfig@legend@RAM@expl{\if@RAM R(A)M $\widehat=$ Researcher (Assistant) Months\else\ Efforts in PM\fi}
\newcommand\wpfig@legend@lead@expl{WP lead efforts \wp@lead@style@explained}
\newcommand\wpfig@legend@caption{{\ifwork@areas Work Areas and \fi}Work Packages}
%    \end{macrocode}
% \end{environment}
% 
%
% \begin{macro}{\wpfig}
%   \ednote{The computation can be distributed much more efficiently (by intermingling the
%   counter advances with the row creation), but this works now}
%    \begin{macrocode}
\newcount\local@count
\newcount\@@@RM\if@RAM\newcount\@@@RAM\fi
\newcount\all@@@RM\if@RAM\newcount\all@@@RAM\fi
\newcommand{\wpfig}[1][]{\setcounter{wpfig@options}{0}\setkeys{wpfig}{#1}
%    \end{macrocode}
% the first thing to do is to build the body of the table programmatically by (globally)
% extending the |\@wp@lines| token register inside a bracket group which locally redefines
% all macros we are using in the extensions, so that they do not get into the way. We
% start this group now. 
%    \begin{macrocode}
{\gdef\@wp@lines{}%initialize
\let\tabularnewline\relax\let\hline\relax\let\lead@style\relax% so they
\let\wa@style\relax\let\wp@style\relax \let\@sw\relax\let\textbf\relax% do not
\let\G@refundefinedtrue=\relax\let\@latex@warning=\relax\let\hyperlink=\relax% bother
\let\pn\relax\let\xspace\relax% us
%    \end{macrocode}
% The code that follows now, could be more elegant, if we had a better way of organizing
% the data, but this works for now, we have four cases: with/without work areas and
% with/without sites. All do something very similar. 
%    \begin{macrocode}
\ifwork@areas
\edef\@@was{\pdataref@safe{all}{wa}{ids}}%
\@for\@@wa:=\@@was\do{% iterate over the work areas
\xdef\@@wa@line{\wa@style{\pdataRef{wa}\@@wa{label}}%
&\wa@style{\@ifundefined{wa@\@@wa @short}{\pdataref{wa}\@@wa{title}}{\pdataref{wa}\@@wa{short}}}%
\ifx\wpfig@type\@true&\wa@style{\pdataref{wa}\@@wa{type}}\fi%
\ifx\wpfig@pages\@true&\wa@style{\pdataref{wa}\@@wa{page}}\fi%
\ifx\wpfig@start\@true&\wa@style{\pdataref{wa}\@@wa{start}}\fi%
\ifx\wpfig@length\@true&\wa@style{\pdataref{wa}\@@wa{len}}\fi%
\ifx\wpfig@end\@true&\wa@style{\pdataref{wa}\@@wa{end}}\fi}
\if@sites
\@for\@site:=\prop@gen@sites\do{%
\edef\@@wps{\pdataref@safe\@@wa{wp}{ids}}%
\local@count 0%
\@for\@@wp:=\@@wps\do{\advance\local@count by \pdataref@num\@@wp\@site{RM}}%
\pdata@def\@@wa\@site{RM}{\the\local@count}%
\xdef\@@wa@line{\@@wa@line&\wa@style{\the\local@count}}%
\if@RAM
\local@count 0%
\@for\@@wp:=\@@wps\do{\advance\local@count by \pdataref@num\@@wp\@site{RAM}}
\pdata@def\@@wa\@site{RAM}{\the\local@count}%
\xdef\@@wa@line{\@@wa@line&\wa@style{\the\local@count}}%
\fi}
\local@count0\relax%
\@for\@site:=\prop@gen@sites\do{\global\advance\local@count by \pdataref@num\@@wa\@site{RM}}%
\xdef\@@wa@line{\@@wa@line &\wa@style{\textbf{\the\local@count}}}
\if@RAM
\local@count0\relax%
\@for\@site:=\prop@gen@sites\do{\global\advance\local@count by \pdataref@num\@@wa\@site{RAM}}%
\xdef\@@wa@line{\@@wa@line &\wa@style{\textbf{\the\local@count}}}
\fi
\else% if@sites
\edef\@@wps{\pdataref@safe{all}{wp}{ids}}%
\xdef\@@wa@line{\@@wa@line&\wa@style{\pdataref{wa}\@@wa{RM}}
\if@RAM&\wa@style{\pdataref{wa}\@@wa{RAM}}\fi}%
\fi% if@sites
\xdef\@wp@lines{\@wp@lines\@@wa@line\tabularnewline\hline}% add the line for the workarea
\edef\@@wps{\pdataref@safe\@@wa{wp}{ids}}%
\@for\@@wp:=\@@wps\do{% iterate over its work packages
\xdef\@@wp@line{\pdataRef{wp}\@@wp{label}%
&\@ifundefined{wp@\@@wp @short}{\pdataref{wp}\@@wp{title}}{\pdataref{wp}\@@wp{short}}%
\ifx\wpfig@type\@true&\pdataref{wp}\@@wp{type}\fi%
\ifx\wpfig@pages\@true&\pdataref{wp}\@@wp{page}\fi%
\ifx\wpfig@start\@true&\pdataref{wp}\@@wp{start}\fi%
\ifx\wpfig@length\@true&\pdataref{wp}\@@wp{len}\fi%
\ifx\wpfig@end\@true&\pdataref{wp}\@@wp{end}\fi}
\if@sites
\@for\@site:=\prop@gen@sites\do{%
\edef\@@lead{\pdataref@safe{wp}\@@wp{lead}}
\edef\@@RM{\ifx\@@lead\@site\lead@style{\pdataref@safe\@@wp\@site{RM}}\else\wp@style{\pdataref@safe\@@wp\@site{RM}}\fi}
\xdef\@@wp@line{\@@wp@line&\@@RM}
\if@RAM
\edef\@@RAM{\ifx\@@lead\@site\lead@style{\pdataref@safe\@@wp\@site{RAM}}\else\wp@style{\pdataref@safe\@@wp\@site{RAM}}\fi}
\xdef\@@wp@line{\@@wp@line&\@@RAM}
\fi}
\local@count0\relax%
\@for\@site:=\prop@gen@sites\do{\global\advance\local@count by \pdataref@num\@@wp\@site{RM}}%
\xdef\@@wp@line{\@@wp@line &\textbf{\the\local@count}}
\if@RAM
\global\local@count0\relax%
\@for\@site:=\prop@gen@sites\do{\global\advance\local@count by \pdataref@num\@@wp\@site{RAM}}%
\xdef\@@wp@line{\@@wp@line &\textbf{\the\local@count}}
\fi% if@RAM
\else% if@sites
\xdef\@@wp@line{\@@wp@line&\wp@style{\pdataref@safe{wp}\@@wp{RM}}}
\if@RAM\xdef\@@wp@line{\@@wp@line&\wp@style{\pdataref@safe{wp}\@@wp{RAM}}}\fi
\fi% if@sites
\xdef\@wp@lines{\@wp@lines\@@wp@line\tabularnewline\hline}}}
%    \end{macrocode}
% Now the case where we do not have work areas.
%    \begin{macrocode}
\else% ifwork@areas
\edef\@@wps{\pdataref@safe{all}{wp}{ids}}%
\@for\@@wp:=\@@wps\do{% iterate over its work packages
\xdef\@@wp@line{\pdataRef{wp}\@@wp{label}%
&\@ifundefined{wp@\@@wp @short}{\pdataref{wp}\@@wp{title}}{\pdataref{wp}\@@wp{short}}
\ifx\wpfig@type\@true&\pdataref{wp}\@@wp{type}\fi%
\ifx\wpfig@pages\@true&\pdataref{wp}\@@wp{page}\fi%
\ifx\wpfig@start\@true&\pdataref{wp}\@@wp{start}\fi%
\ifx\wpfig@length\@true&\pdataref{wp}\@@wp{len}\fi%
\ifx\wpfig@end\@true&\pdataref{wp}\@@wp{end}\fi}
\if@sites
\@for\@site:=\prop@gen@sites\do{%
\edef\@@lead{\pdataref@safe{wp}\@@wp{lead}}
\edef\@@RM{\ifx\@@lead\@site\lead@style{\pdataref@safe\@@wp\@site{RM}}\else\wp@style{\pdataref@safe\@@wp\@site{RM}}\fi}
\xdef\@@wp@line{\@@wp@line&\@@RM}
\if@RAM
\edef\@@RAM{\ifx\@@lead\@site\lead@style{\pdataref@safe\@@wp\@site{RAM}}\else\wp@style{\pdataref@safe\@@wp\@site{RAM}}\fi}
\xdef\@@wp@line{\@@wp@line&\wp@style\@@RAM}
\fi}
\global\local@count0\relax%
\@for\@site:=\prop@gen@sites\do{\global\advance\local@count by \pdataref@num\@@wp\@site{RM}}%
\xdef\@@wp@line{\@@wp@line &\textbf{\the\local@count}}
\if@RAM
\global\local@count0\relax%
\@for\@site:=\prop@gen@sites\do{\global\advance\local@count by \pdataref@num{#1}\@site{RAM}}%
\xdef\@@wp@line{\@@wp@line &\textbf{\the\local@count}}
\fi
\else% if@sites
\xdef\@@wp@line{\@@wp@line&\wp@style{\pdataref@safe{wp}\@@wp{RM}}}
\if@RAM\xdef\@@wp@line{\@@wp@line&\wp@style{\pdataref@safe{wp}\@@wp{RAM}}\fi}
\fi% if@sites
\xdef\@wp@lines{\@wp@lines\@@wp@line\tabularnewline\hline}}
\fi%ifwork@areas
%    \end{macrocode}
% Now we compute the totals lines in the |\@totals| macros; again there are four cases to
% consider
%    \begin{macrocode}
\gdef\@totals{}
\ifwork@areas
\if@sites
\@for\@site:=\prop@gen@sites\do{% iterate over the sites
\@@@RM=0\if@RAM\@@@RAM=0\fi
\edef\@@was{\pdataref@safe{all}{wa}{ids}}%
\@for\@@wa:=\@@was\do{% iterate over the work areas
\edef\@@wps{\pdataref@safe\@@wa{wp}{ids}}%
\@for\@@wp:=\@@wps\do{% iterate over the work packages
\advance\@@@RM by \pdataref@num\@@wp\@site{RM}%
\if@RAM\advance\@@@RAM by \pdataref@num\@@wp\@site{RAM}\fi}}
\pdata@def{all}\@site{RM}{\the\@@@RM}\if@RAM\pdata@def{all}\@site{RAM}{\the\@@@RAM}\fi
\advance\all@@@RM by \the\@@@RM\if@RAM\advance\all@@@RAM by \the\@@@RAM\fi
\xdef\@totals{\@totals & \textbf{\the\@@@RM}\if@RAM& \textbf{\the\@@@RAM}\fi}}
\xdef\@totals{\@totals & \textbf{\the\all@@@RM}\if@RAM&\textbf{\the\all@@@RAM}\fi}
\pdata@def{all}{total}{RM}{\the\all@@@RM}\if@RAM\pdata@def{all}{total}{RAM}{\the\all@@@RAM}\fi
\else% if@sites
\@@@RM=0\if@RAM\@@@RAM=0\fi
\edef\@@was{\pdataref@safe{all}{wa}{ids}}%
\@for\@@wa:=\@@was\do{\edef\@@wps{\pdataref@safe\@@wa{wp}{ids}}%
\@for\@@wp:=\@@wps\do{% iterate over the work packages
\advance\@@@RM by \pdataref@num{wp}\@@wp{RM}%
\if@RAM\advance\@@@RAM by \pdataref@num{wp}\@@wp{RAM}\fi}}
\pdata@def{all}{total}{RM}{\the\@@@RM}\if@RAM\pdata@def{all}{total}{RAM}{\the\@@@RAM}\fi
\xdef\@totals{&\the\@@@RM\if@RAM &\the\@@@RAM\fi}
\fi% if@sites
\else%i.e. no work@areas
\if@sites
\@for\@site:=\prop@gen@sites\do{%iterate over the sites
\@@@RM=0\if@RAM\@@@RAM=0\fi%
\edef\@@wps{\pdataref@safe{all}{wp}{ids}}%
\@for\@@wp:=\@@wps\do{% iterate over the work packages
\advance\@@@RM by \pdataref@num\@@wp\@site{RM}%
\if@RAM\advance\@@@RAM by \pdataref@num\@@wp\@site{RAM}\fi}
\pdata@def{all}\@site{RM}{\the\@@@RM}\if@RAM\pdata@def{all}\@site{RAM}{\the\@@@RAM}\fi
\xdef\@totals{\@totals & \textbf{\the\@@@RM}\if@RAM& \textbf{\the\@@@RAM}\fi}
\advance\all@@@RM by \the\@@@RM\if@RAM\advance\all@@@RAM by \the\@@@RAM\fi}
\xdef\@totals{\@totals &\textbf{\the\all@@@RM}\if@RAM&\textbf{\the\all@@@RAM}\fi}
\pdata@def{all}{total}{RM}{\the\all@@@RM}\if@RAM\pdata@def{all}{total}{RAM}{\the\all@@@RAM}\fi
\else% if@sites
\@@@RM=0\if@RAM\@@@RAM=0\fi
\edef\@@wps{\pdataref@safe{all}{wp}{ids}}%
\@for\@@wp:=\@@wps\do{% iterate over the work packages
\advance\@@@RM by \pdataref@num{wp}\@@wp{RM}%
\if@RAM\advance\@@@RAM by \pdataref@num{wp}\@@wp{RAM}\fi}
\pdata@def{all}{total}{RM}{\the\@@@RM}\if@RAM\pdata@def{all}{total}{RAM}{\the\@@@RAM}\fi
\xdef\@totals{&\the\@@@RM\if@RAM &\the\@@@RAM\fi}
\fi% if@sites
\fi
%    \end{macrocode}
% And we finally have a line for the intended totals which we use in draft mode. 
%    \begin{macrocode}
\gdef\intended@totals{}\gdef\requested@totals{}
\if@sites
\@for\@site:=\prop@gen@sites\do{
\xdef\intended@totals{\intended@totals&\textbf{\pdataref@safe{site}\@site{intendedRM}}}
\xdef\requested@totals{\requested@totals&\pdataref@safe{site}\@site{reqPM}}
\if@RAM\xdef\intended@totals{\intended@totals&\textbf{\pdataref@safe{site}\@site{intendedRAM}}}\fi}
\if@RAM\xdef\intended@totals{\intended@totals&&}\else%
\xdef\intended@totals{\intended@totals&}%
\xdef\requested@totals{\requested@totals&}%
\fi
\else% if@sites
\xdef\intended@totals{\intended@totals&\textbf{\pdataref@safe{all}{intended}{RM}}}
\if@RAM\xdef\intended@totals{\intended@totals&\textbf{\pdataref@safe{all}{intended}{RAM}}}\fi
\fi}% if@sites
%    \end{macrocode}
% finally, we make all of this into a figure, computing the colspan of the the legend
% cells for the totals via |\local@count| from the optional columns.
%    \begin{macrocode}
\local@count\thewpfig@options\advance\local@count by 2
\begin{wp@figure}
\@wp@lines\hline%
\multicolumn{\the\local@count}{|c|}{\prop@legend@totals}\@totals\\\hline%
\ifsubmit\else%
\ifx\prop@gen@topdownPM\@true%
\multicolumn{\the\local@count}{|c|}{\prop@legend@intendedtotals}\intended@totals\\\hline%
\fi% topdownPM
\ifx\prop@gen@botupPM\@true%
\multicolumn{\the\local@count}{|c|}{\prop@legend@requestedtotals}\requested@totals\\\hline%
\fi% botupPM
\fi% submit
\end{wp@figure}}
%    \end{macrocode}
% and now multilinguality support 
%    \begin{macrocode}
\newcommand\prop@legend@totals{\textbf{totals}}
\newcommand\prop@legend@intendedtotals{\textbf{intended totals}}
\newcommand\prop@legend@requestedtotals{\textbf{requested totals}}
%    \end{macrocode}
% \end{macro}
% 
% \subsection{Gantt Charts}\label{sec:impl:gantt}
%
% Gantt Charts are done with help of the the |tikz| package. The |gantt|
% environments pick up on the declared duration of the proposal in months stored in the
% |\prop@gen@months| macro.
% 
% We define the keys for Gantt tables
%    \begin{macrocode}
\newif\ifgantt@draft\gantt@draftfalse
\newif\ifgantt@miles\gantt@milesfalse
\define@key{gantt}{xscale}{\def\gantt@xscale{#1}}
\define@key{gantt}{yscale}{\def\gantt@yscale{#1}}
\define@key{gantt}{step}{\def\gantt@step{#1}}
\define@key{gantt}{size}{\def\gantt@size{#1}}
\define@key{gantt}{draft}[true]{\ifsubmit\else\gantt@drafttrue\fi}
\define@key{gantt}{milestones}[true]{\gantt@milestrue}
%    \end{macrocode}
% Then we define an auxiliary function that provides defaults for these keys and sets the
% internal macros.
%    \begin{macrocode}
\def\gantt@set#1{\gantt@draftfalse\def\gantt@xscale{1}\def\gantt@yscale{.35}\def\gantt@step{3}
\setkeys{gantt}{#1}}
%    \end{macrocode}
%
% Finally, the Gantt Chart environment itself.
%
% \begin{environment}{gantt}
%   The |gantt[|\meta{keyvals}|]{|\meta{height}|}| environment sets up the grid and legend
%   for a gantt chart. The grid is |\prop@gen@months| wide and \meta{height} high. 
%    \begin{macrocode}
\newenvironment{gantt}[2][]
{\gantt@set{#1}\gdef\gantt@height{#2}
\def\@test{\prop@gen@months@default}
\ifx\@test\prop@gen@months
\ClassError{proposal}{Need overall project months to draw gantt
    chart - expect trouble;\MessageBreak specify
    \protect\begin{proposal}[...,months=??,...] to fix}\fi
\@ifundefined{gantt@size}{}{\csname\gantt@size\endcsname}
\newdimen\gantt@ymonths
\gantt@ymonths=\gantt@height cm
\advance\gantt@ymonths by .8cm
\begin{tikzpicture}[xscale=\gantt@xscale,yscale=\gantt@yscale]}
{\draw[xstep=\gantt@step,very thin] (0,0) grid (\prop@gen@months,\gantt@height);
\foreach \x in {0,\gantt@step,...,\prop@gen@months} \node at (\x,\gantt@ymonths) {\x};
\ifgantt@miles
\newdimen\gantt@ymiles\gantt@ymiles=\gantt@height cm
\advance\gantt@ymiles by 2cm
\newdimen\gantt@ymiles@top\gantt@ymiles@top=\gantt@height cm
%\advance\gantt@ymiles@top by 2cm
\edef\@@miles{\pdataref@safe{all}{mile}{ids}}
\@for\@I:=\@@miles\do{%
\edef\@@month{\pdataref@safe{mile}{\@I}{month}}
\draw[very thick,blue] (\@@month,\gantt@ymiles@top) -- (\@@month,0);
\node[blue] at (\@@month,\gantt@ymiles) {\pdataref{mile}{\@I}{label}};}
\fi %gantt@miles
\end{tikzpicture}}
%    \end{macrocode}
% \end{environment}
%
% \begin{macro}{\@action}
%   In this we have used the macro that does the actual painting.
%   |\@action{|\meta{name}|}{|\meta{line}|}{|\meta{start}|}{|\meta{len}|}{|\meta{force}|}|
%   creates a gantt node with name \meta{name} in line \meta{line} starting at month
%   \meta{month} with length \meta{len} that is \meta{force} thick.
%    \begin{macrocode}
\newdimen\gantt@ymid\newdimen\gantt@yinc\newdimen\gantt@xend
\newcommand{\@action}[6][]{\def\@test{#1}%
\ifx\@test\@empty\def\@@color{ganttgray}\else\def\@@color{#1}\fi
\gantt@ymid=#3 cm\gantt@yinc=\gantt@yscale cm
\gantt@xend=#4 cm\advance\gantt@xend by #5 cm
\advance\gantt@ymid by \gantt@yinc
\fill[\@@color] (#4,#3) rectangle +(#5,#6);
\node (#2@left) at (#4,\gantt@ymid) {};
\node (#2@right) at (\gantt@xend,\gantt@ymid) {};}
%    \end{macrocode}
% \end{macro}
% 
% \begin{macro}{\@dependency}
%    \begin{macrocode}
\def\@dependency#1#2{\draw[->,line width=2pt,color=red] (#1@right) -- (#2@left);}
%    \end{macrocode}
% \end{macro}
% 
% \begin{macro}{\gantt@compute@effort}
%   A helper function that updates the dimension |\gantt@effort| according to whether the
%   counter |\gantt@month| is in the range. It is used in |\gantt@chart|
%    \begin{macrocode}
\newcommand\gantt@compute@effort[3]{% start, len, force
  \@@e=#1\advance\@@e by #2
  \ifnum\thegantt@month<#1\else
  \ifnum\thegantt@month<\@@e
  \gantt@plus=#3cm\advance\gantt@effort by \gantt@plus\fi\fi}
%    \end{macrocode}
% \end{macro}
%
% \begin{macro}{\ganttchart}
%   This macro iterates over the work areas, their work packages, and finally their work
%   phases to use the internal macro |\@action|. All of this in the gantt setting.
%    \begin{macrocode}
\newcommand{\ganttchart}[1][]{\begin{figure}[ht]\centering
\gantt@set{#1}
\def\gantt@wps{\pdataref@num{all}{wp}{count}}
\begin{gantt}[#1]{\gantt@wps}
 \newcounter{taskwps}\newcount\@@line
 \edef\@@was{\pdataref@safe{all}{wa}{ids}}
 \ifwork@areas
 \@for\@@wa:=\@@was\do{% iterate over work areas
   \edef\@@wps{\pdataref@safe\@@wa{wp}{ids}}
   \@for\@@wp:=\@@wps\do{% iterate over work packages
     \stepcounter{taskwps}
     \@@line=\gantt@wps\advance\@@line by -\thetaskwps
     \edef\@@tasks{\pdataref@safe\@@wp{task}{ids}}
     \node at (-1/\gantt@xscale,\@@line) [above=-2pt] {\pdataRef{wp}\@@wp{label}};
     \edef\@@wphases{\pdataref@safe{wp}\@@wp{wphases}}
     \@for\@@ft:=\@@wphases\do{%wp-level work phases
       \decode@wphase\@@ft
       \@action\@@wp\@@line\wphase@start\wphase@len\wphase@force}
     \@for\@@task:=\@@tasks\do{% tasks
       \edef\@@wphases{\pdataref@safe{task}\@@task{wphases}}
       \@for\@@ft:=\@@wphases\do{%task-level work phases
         \decode@wphase\@@ft
         \@action\@@task\@@line\wphase@start\wphase@len\wphase@force}}}}
 \else% ifwork@areas false
 \edef\@@wps{\pdataref@safe{all}{wp}{ids}}
 \@for\@@wp:=\@@wps\do{% iterate over work packages
   \stepcounter{taskwps}
   \@@line=\gantt@wps\advance\@@line by -\thetaskwps
   \edef\@@tasks{\pdataref@safe\@@wp{task}{ids}}
   \node at (-1/\gantt@xscale,\@@line) [above=-2pt] {\pdataRef{wp}\@@wp{label}};
   \edef\@@wphases{\pdataref@safe{wp}\@@wp{wphases}}
   \@for\@@ft:=\@@wphases\do{%iterate over the wp-level work phases
     \decode@wphase\@@ft
     \@action\@@wp\@@line\wphase@start\wphase@len\wphase@force}
   \@for\@@task:=\@@tasks\do{% task-level work phases
     \edef\@@wphases{\pdataref@safe{task}\@@task{wphases}}
     \@for\@@ft:=\@@wphases\do{%iterate over the task-level work phases
       \decode@wphase\@@ft
       \@action\@@task\@@line\wphase@start\wphase@len\wphase@force}}}
 \fi% ifwork@areas end
 \edef\@@deps{\pdataref@safe{all}{task}{deps}}
 \@for\@@dep:=\@@deps\do{%
   \@dependency{\pdataref@safe{taskdep}\@@dep{from}}{\pdataref@safe{taskdep}\@@dep{to}}}
%    \end{macrocode}
% The next piece of code generates the effort sum table in draft mode
%    \begin{macrocode}
 \ifgantt@draft
    \newcounter{gantt@month}
    \newcount\@@e\newdimen\gantt@effort\newdimen\gantt@plus
    \@whilenum\thegantt@month<\prop@gen@months\do{% step over months
      \gantt@effort=0cm
      \ifwork@areas
      \edef\@@was{\pdataref@safe{all}{wa}{ids}}
      \@for\@@wa:=\@@was\do{% iterate over work areas
        \edef\@@wps{\pdataref@safe\@@wa{wp}{ids}}
        \@for\@@wp:=\@@wps\do{% iterate over work packages
          \edef\@@wphases{\pdataref@safe{wp}\@@wp{wphases}}
          \@for\@@ft:=\@@wphases\do{%iterate over the wp-level work phases
            \decode@wphase\@@ft
            \gantt@compute@effort\wphase@start\wphase@len\wphase@force}
          \edef\@@tasks{\pdataref@safe\@@wp{task}{ids}}
          \@for\@@task:=\@@tasks\do{% iterate over tasks
          \edef\@@wphases{\pdataref@safe{task}\@@task{wphases}}
          \@for\@@ft:=\@@wphases\do{%iterate over the wp-level work phases
            \decode@wphase\@@ft
            \gantt@compute@effort\wphase@start\wphase@len\wphase@force}}}}
      \fill[ganttgray]  (\thegantt@month,-5) rectangle +(1,\gantt@effort);
      \else% ifwork@areas
      \edef\@@wps{\pdataref@safe{all}{wp}{ids}}
      \@for\@@wp:=\@@wps\do{% iterate over work packages
          \edef\@@wphases{\pdataref@safe{wp}\@@wp{wphases}}
          \@for\@@ft:=\@@wphases\do{%iterate over the wp-level work phases
            \decode@wphase\@@ft
            \gantt@compute@effort\wphase@start\wphase@len\wphase@force}
          \edef\@@tasks{\pdataref@safe\@@wp{task}{ids}}
          \@for\@@task:=\@@tasks\do{% iterate over tasks
          \edef\@@wphases{\pdataref@safe{task}\@@task{wphases}}
          \@for\@@ft:=\@@wphases\do{%iterate over the wp-level work phases
            \decode@wphase\@@ft
            \gantt@compute@effort\wphase@start\wphase@len\wphase@force}}}
      \fill[ganttgray]  (\thegantt@month,-5) rectangle +(1,\gantt@effort);
      \fi% ifwork@areas
      \stepcounter{gantt@month}}
   \fi% ifgantt@draft
  \end{gantt}
  \caption{\gantt@caption}\label{fig:gantt}
\end{figure}\footnotetext\gantt@footnote}
%    \end{macrocode}
% now the multilingual support
%    \begin{macrocode}
\newcommand\gantt@caption@main{Gantt Chart: Overview Work Package Activities}
\newcommand\gantt@caption@lower{lower bar shows the overall effort \if@RAM (RM only) \fi per month}
\newcommand\gantt@caption{\gantt@caption@main\ifgantt@draft\xspace
  -- \gantt@caption@lower\fi}
\newcommand\gantt@footnote{Bars shown at reduced height (e.g. 50\%) indicate reduced
  intensity during  that work phase  (e.g. to 50\%).}
%    \end{macrocode}
% \end{macro}
%
% \begin{macro}{\gantttaskchart}
%   This macro is a variant of |\ganttchart|, but it shows the tasks consecutively, as is
%   useful for EU projects\ednote{this should be incorporated with the gantt chart above,
%   but I am currently to scared to do it so close to the deadline}
%    \begin{macrocode}
\newcommand{\gantttaskchart}[1][]{\begin{figure}[hbtp]\centering\gantt@set{#1}
\newcounter{gantt@all@tasks}%
\setcounter{gantt@all@tasks}{\pdataref@num{all}{task}{count}}
\addtocounter{gantt@all@tasks}{\pdataref@num{all}{wp}{count}}
\begin{gantt}[#1]{\thegantt@all@tasks}
  \newcounter{gantt@tasks}\newcount\@@line
  \edef\@@wps{\pdataref@safe{all}{wp}{ids}}
   \@for\@@wp:=\@@wps\do{% iterate over work packages
     \stepcounter{gantt@tasks}
%     \@action[white]{}\@@line0{48}1
     \edef\@@tasks{\pdataref@safe\@@wp{task}{ids}}
     \@for\@@task:=\@@tasks\do{% iterate over the tasks
       \stepcounter{gantt@tasks}
       \@@line=\thegantt@all@tasks\advance\@@line by -\thegantt@tasks
       \node at (-.5/\gantt@xscale,\@@line) [above=-2pt] {{\footnotesize\taskreflong\@@wp\@@task}};
       \edef\@@wphases{\pdataref@safe{task}\@@task{wphases}}
       \@for\@@ft:=\@@wphases\do{%iterate over the task-level work phases
         \decode@wphase\@@ft
         \@action\@@task\@@line\wphase@start\wphase@len\wphase@force
      }}}% end all iterations
    \end{gantt}
    \caption{\gantt@caption@main{} -- \emph{\gantt@footnote}}\label{fig:gantt}
\end{figure}}
%    \end{macrocode}
% \end{macro}
%
% \subsection{Coherence}\label{sec:impl:coherence}
%
% \begin{macro}{\j*}
%    \begin{macrocode}
\newcommand\jpub{\textcolor{\prop@link@color}{\textbf{\Large{$\star$}}}}
\newcommand\jpro{\textcolor{\prop@link@color}{\textbf{\Large{$\bullet$}}}}
\newcommand\jsoft{\textcolor{\prop@link@color}{\textbf{@}}}
\newcommand\jorga{\textcolor{\prop@link@color}{\textbf{\Large{$\circ$}}}}
\newcommand\jsup{\textcolor{\prop@link@color}{\textbf{\smiley}}}
%    \end{macrocode}
% \end{macro}
%
% \begin{macro}{\add@joint}
%   |\add@joint{|\meta{first}|}{|\meta{second}|}{|\meta{sym}|}| adds \meta{sym} to the the
%   |\coherence@|\meta{first}|@|\meta{second} macro for the coherence table.
%    \begin{macrocode}
\newcommand\add@joint[3]{\@ifundefined{coherence@#1@#2}%
{\@namedef{coherence@#1@#2}{#3}}%
{\expandafter\g@addto@macro\csname coherence@#1@#2\endcsname{#3}}}
%    \end{macrocode}
% \end{macro}
%
% \begin{macro}{\prop@joint}
%   This iterates over a comma-separated list of names and makes the necessary entries
%   into the coherence table.
%    \begin{macrocode}
\newcommand\prop@joint[2]{\@for\@first:=#2\do{%
\@for\@second:=#2\do{\ifx\@first\@second\else\add@joint\@first\@second{#1}\fi}}}
%    \end{macrocode}
% \end{macro}
%
% \begin{macro}{\joint*}
%   Now, some instances that use these.
%    \begin{macrocode}
\newcommand\jointproj[1]{\prop@joint\jpro{#1}}
\newcommand\jointpub[1]{\prop@joint\jpub{#1}}
\newcommand\jointorga[1]{\prop@joint\jorga{#1}}
\newcommand\jointsoft[1]{\prop@joint\jsoft{#1}}
\newcommand\jointsup[1]{\prop@joint\jsup{#1}}
%    \end{macrocode}
% \end{macro}
% 
% \begin{macro}{\coherencematrix}
%    \begin{macrocode}
\newcommand{\coherencematrix}{
{\let\tabularnewline\relax\let\hline\relax\let\site\relax% so they do
 \let\@sw\relax\let\jpub\relax\let\jpro\relax\let\jorga\relax% not bother
 \let\jsoft\relax\let\jsup\relax\let\cellcolor\relax% us
\gdef\@ct@head{}%
\@for\@site:=\prop@gen@sites\do{\xdef\@ct@head{\@ct@head%
&\ifx\cht@swsites\@true\@sw{\site{\@site}}\else\site{\@site}\fi}}%
\gdef\@ct@lines{\@ct@head\tabularnewline\hline\hline} %initialize with head line
\@for\@site:=\prop@gen@sites\do{\xdef\@ct@line{\site{\@site}}%
  \@for\@@site:=\prop@gen@sites\do{%
    \xdef\@ct@line{\@ct@line&\ifx\@site\@@site{\cellcolor{wagray}{}}\fi%
      \@ifundefined{coherence@\@site @\@@site}{}{\@nameuse{coherence@\@site @\@@site}}}}%
  \xdef\@ct@lines{\@ct@lines\@ct@line\tabularnewline\hline}}}%
\begin{tabular}{|l||*{\the@site}{c|}}\hline%
\@ct@lines\hline%
joint&\multicolumn{\the@site}{l|}{\jpub $\hat=$ publication, \jpro $\hat=$ project,
       \jorga $\hat=$ organization, \jsoft $\hat=$ software/resource dev,
       \jsup $\hat=$ supervision}\\\hline
\end{tabular}}
%    \end{macrocode}
% \end{macro}
% 
% \begin{macro}{\coherencetable}
%    \begin{macrocode}
\newskip\@bigflushglue \@bigflushglue = -100pt plus 1fil
\def\bigcenter{\trivlist \bigcentering\item\relax}
\def\bigcentering{\let\\\@centercr\rightskip\@bigflushglue%
\leftskip\@bigflushglue
\parindent\z@\parfillskip\z@skip}
\def\endbigcenter{\endtrivlist}
\define@key{coherencetable}{swsites}[true]{\def\cht@swsites{#1}}
\define@key{coherencetable}{stretch}{\def\cht@stretch{#1}}
\newcommand\coherencetable[1][]{%
\def\cht@swsites{false}%
\def\cht@stretch{1}%
\setkeys{coherencetable}{#1}%
\begin{table}[ht]%
\small\setlength{\tabcolsep}{.5em}%
\renewcommand{\arraystretch}{\cht@stretch}%
\begin{bigcenter}%
\coherencematrix%
\end{bigcenter}%
\caption{\coherence@caption}\label{tab:collaboration}
\end{table}}
%    \end{macrocode}
% now the multilinguality support
%    \begin{macrocode}
\newcommand\coherence@caption{Previous Collaboration between {\pn} members}
%    \end{macrocode}
% \end{macro}
%
%\subsection{Relevant Papers \& References}\label{sec:imple:papers}
%
%   We first define a bibLaTeX bibliography heading that does not create headers, we need
%   it somewhere. 
%    \begin{macrocode}
\defbibheading{empty}{}
%    \end{macrocode}
% We define an internal macro that prints a publication list of a given {bib\TeX} entry
% type and title for convenience. It also adds a |notype=| to the token register
% |\prop@rl| to deal with the unclassified entries from the list. 
%    \begin{macrocode}
\newif\if@allpapers\@allpaperstrue
\newcommand\prop@ppl[3][]{\@allpapersfalse\message{ppl processing: #2}%
\printbibliography[heading=subbibliography,type=#2,title=#3#1]%
\@ifundefined{prop@rl}{\xdef\prop@rl{#2}}{\xdef\prop@rl{\prop@rl, #2}}}
%    \end{macrocode}
% The following code does not work yet, it would have been nice to be able to just add a
% key |unclassified| to catch the unclassified ones. I guess we just have to issue a
% warning instead. 
%    \begin{macrocode}
\newcommand\prop@prl[1]{\message{unclassified: #1}%
\printbibliography[heading=subbibliography,title=Unclassified,#1]}%
\define@key{paperlist}{unclassified}[true]{\message{unclass: \prop@rl}\prop@prl\prop@rl}
%    \end{macrocode}
% with this, we define a couple of keys that generate 
%    \begin{macrocode}
\define@key{paperlist}{articles}[true]{\prop@ppl{article}{Articles}}
\define@key{paperlist}{chapters}[true]{\prop@ppl{inbook}{Book Chapters}}
\define@key{paperlist}{confpapers}[true]{\prop@ppl[,keyword=conference]{inproceedings}{Conference Papers}}
\define@key{paperlist}{wspapers}[true]{\prop@ppl[,notkeyword=conference]{inproceedings}{Workshop Papers}}
\define@key{paperlist}{theses}[true]{\prop@ppl{thesis}{Theses}}
\define@key{paperlist}{submitted}[true]{\prop@ppl[,keyword=submitted]{unpublished}{Submitted}}
\define@key{paperlist}{books}[true]{\prop@ppl{book}{Monographs}}
\define@key{paperlist}{techreports}[true]{\prop@ppl{techreport}{Technical Reports}}
%    \end{macrocode}
%
% \begin{macro}{featured}
% We introduce a new bibLaTeX category |featured| for those papers that were already
% mentioned in |\prop@paperlist| and the macros defined from it. 
%    \begin{macrocode}
\DeclareBibliographyCategory{featured}
%    \end{macrocode}
% \end{macro}
%
% \begin{macro}{\prop@paperlist}
%   |\prop@paperlist{|\meta{keys}|}{|\meta{refs}|}| generates a paper list from a list
%   \meta{keys} of bibliography keys. It makes some local adaptions to the appearance of
%   the bibliography, and then adds \meta{refs} to the citable papers marks them as
%   |featured|. Then it uses |\printbibliography| to make a bibliography of the cited
%   papers. Note that these are not cited again in the main bibliography\ednote{MK: we may
%   want to make this optional controlled by a package option eventually.}
%    \begin{macrocode}
\newcommand\prop@paperlist[2][]{%
\let\biboldfont\bibfont%
\renewcommand{\bibfont}{\footnotesize}%
\renewcommand{\baselinestretch}{.9}%
\nocite{#2}\def\do##1{\addtocategory{featured}{##1}}\docsvlist{#2}%
\setkeys{paperlist}{#1}
\@ifundefined{prop@rl}{}{\@latex@warning{some papers are not classified!}}
\if@allpapers\printbibliography[category=featured,heading=empty]\fi%
\let\bibfont\biboldfont}
%    \end{macrocode}
% \end{macro}
%
% We define the |warnpubs| heading constructor. 
%
%    \begin{macrocode}
\def\prop@warnpubs@message{Many of the proposers' publications are online at one of the following URIs:}
\def\prop@warnpubs@title{References}
\defbibheading{warnpubs}{\section*{\prop@warnpubs@title}%
  \@ifundefined{prop@gen@pubspages}
 {\@latex@warning{No publication pages specified;
                   use the pubspage key in the proposal environment!}}
  {\prop@warnpubs@message%
 \@for\@I:=\prop@gen@pubspages\do{\par\noindent\csname\@I\endcsname}}}
%    \end{macrocode}
% 
% Finally, we tweak {bib\LaTeX} to not give DOIs and URLS at the same time. 
%   \begin{macrocode}
\renewbibmacro*{event+venue+date}{}
\renewbibmacro*{doi+eprint+url}{%
  \iftoggle{bbx:doi}
    {\printfield{doi}\iffieldundef{doi}{}{\clearfield{url}}}
    {}%
  \newunit\newblock
  \iftoggle{bbx:eprint}
    {\usebibmacro{eprint}}
    {}%
  \newunit\newblock
  \iftoggle{bbx:url}
    {\usebibmacro{url+urldate}}
    {}}
%</sty>
%    \end{macrocode}
%
% \subsection{Miscellaneous}\label{sec:impl:misc}
% 
% \begin{macro}{\signatures}
%    \begin{macrocode}
%<*pdata>
\newcommand{\signatures}[1]{\section{#1}
\qquad\number\day. \number\month. \number\year\\[6ex]
\strut\qquad Date\hfill\@for\@p:=\prop@gen@PIs\do{%
\wa@ref3{person}\@p{personaltitle}~\wa@ref3{person}\@p{name}\hfill}}
%    \end{macrocode}
% \end{macro}
% 
% 
% \begin{macro}{\@dmp}
%   The |\@dmp| macro shows metadata information about the keys in the margin if
%   |\keystrue| is specified. This is a debugging tool.
%    \begin{macrocode}
\def\@dmp#1{\ifkeys\marginpar{#1}\fi}
%    \end{macrocode}
% \end{macro}
%
% \begin{macro}{\euro}
%    \begin{macrocode}
\renewcommand\euro{\officialeuro\xspace}
%</pdata>
%    \end{macrocode}
% \end{macro}
% \Finale
\endinput
% LocalWords:  proposal dtx xkohlhase areastrue svninfo noworkareas twoside emph gitinfo
% LocalWords:  areasfalse wa textcomp amssymb url graphicx colortbl xcolor RO OJBref sl
% LocalWords:  fancyhdr keyval eurosym tikz calc USenglish ngerman linkcolor ne WPref gen
% LocalWords:  citecolor urlcolor colorlinks pagecolor breaklinks bookmarksopen WPref doi
% LocalWords:  hyperref LaTeX workpackage PIs thema acrolong anzungsantrag im Mitarbeiter
% LocalWords:  Einzelverfahren tabline Allgemeine Angaben Antragsteller worktel ifsvninfo
% LocalWords:  bitabline personaltitle academictitle birthdate Dienstanschrift Hilfskraft
% LocalWords:  workaddress atabline workfax Privatanschrift privaddress privtel ifgitinfo
% LocalWords:  dt Kennwort Fachgebiet und Arbeitsrichtung Voraussichtliche wa fancyfoot
% LocalWords:  Gesamtdauer Antragszeitraum Zusammenfassung workarea wp wps num prop@gen
% LocalWords:  refundefinedtrue lightgray gantt xstep Unterschriften Id areas acro Lutz
% LocalWords:  submit hide ed public article paper today array comment wide RE tweakable
% LocalWords:  babel true bookmarks blue LO LE page of prop sites key site RM hypertarget
% LocalWords:  def id title acronym discipline start months Proposal Erg by cls endinput
% LocalWords:  empty em name affiliation Applicants Work Address email The pn Hartenstein
% LocalWords:  curriculum vita the applicants are enclosed appendix reference Schr doctex
% LocalWords:  Research area and field work Scientific Fields Anticipated write botupPM
% LocalWords:  duration Application period Summary short lead Package label task botupMP
% LocalWords:  tasks deps number line Effort target ids count warning at pdata swsites
% LocalWords:  for undefined safe zero one two three four five six seven eight swsites
% LocalWords:  nine ten eleven twelve style lines totals Packages xscale yscale  TeXLive
% LocalWords:  step set gray very thin grid workphase len force phase length to endhead
% LocalWords:  taskdep from rectangle left right width pt color red above dep jointsub
% LocalWords:  Overview Activities Signatures Date paralist ymonths ymid yinc jointsoft
% LocalWords:  xend effort month iffalse eustrep.dtx fileversion maketitle Antr jointsoft
% LocalWords:  tableofcontents DescribeEnv ifpublic ifpublic compactitem wpfig jointsup
% LocalWords:  wpfig workplan workplan bookeeping wissenchaftliche tasklist dmp jointsup
% LocalWords:  tasklist taskin taskref taskref tasktref tasktref localtaskref jsoft jsoft
% LocalWords:  localtaskref ganttchart biblatex ifx dfgproposal.dtx newpage pn jsup jsup
% LocalWords:  bibtex proposal1-blx.aux pdataref WPtref WPtref WAref WAref WAtref wasysym
% LocalWords:  WAtref protectmacro thebibliography bibitem newblock newif inst optimized
% LocalWords:  ifsubmit submitfalse publicfalse ifkeys keysfalse ifwork RAMtrue sec:intro
% LocalWords:  submittrue publictrue excludecomment RAMfalse keystrue sfdefault stylized
% LocalWords:  renewcommand familydefault helvet includecomment subsubsections organizing
% LocalWords:  setcounter secnumdepth ifofpage ofpagefalse fancyhead thepage sw 5,step
% LocalWords:  pagestyle fancyplain sitesfalse sitestrue ifundefined gdef allwp a,b,c
% LocalWords:  fundedperiod reportperiod textbf newenvironment titlepage auxout itemize
% LocalWords:  thispagestyle newcounter stepcounter tocdepth ofpagetrue setkeys 2pt,color
% LocalWords:  finalreport orderzeitraum thewa thewp addtocounter bigskip pws specialized
% LocalWords:  textsf smallskip ignorespaces subsubsection  theallwp req pnlong centering
% LocalWords:  compactenum compactenum thetask newcount thegantt refsection tl emphasized
% LocalWords:  nocite defbibheading ourpubs printbibliography expandafter ifnum fig:gantt
% LocalWords:  csname endcsname count ifcase cellcolor wptable widehat qquad ct 2,title
% LocalWords:  tabularnewline wplist ifgantt drafttrue RMdisclaimer draftfalse fig:wplist
% LocalWords:  RMdisclaimerfalse RMdisclaimertrue newdimen taskwps thetaskwps reposistory
% LocalWords:  whilenum hfill marginpar texteuro officialeuro texttt pnlong wpg docsvlist
% LocalWords:  roject xspace internallly OBJref OBJtref OJBtref ldots jointpub specialize
% LocalWords:  pdatacount pdatacount coherencematrix coherencematrix jointpub gitinfo2:on
% LocalWords:  jointproj jointproj jointorga jointorga jpub jpub jproj jproj rl eprint
% LocalWords:  jorga coherencetable coherencetable vspace ifdelivs delivsfalse auto,style
% LocalWords:  delivstrue boxedminipage ednotes thempfootnote mpfootnote xdef visualizing
% LocalWords:  csquotes eso-foot ednoteshape footnotesize importfrom readpdata initialize
% LocalWords:  pubspage newcommand jobname.pdata jobname compromized warnpubs final,today
% LocalWords:  maxbibnames reportfalse reporttrue noindent goodbreak theobjective bbx:doi
% LocalWords:  deliv ednote wpheadertable wptitle wprm hline theprop newwrite thick,blue
% LocalWords:  delivs openout deliverables themilestone wpdelivs textwidth lec iftoggle
% LocalWords:  compactdesc wadelivs hfil nobreak hbox leadsto delivref dissem cht@swsites
% LocalWords:  delivtref dissem wpdeliv thedeliv textit wadeliv verif verif rgb newunit
% LocalWords:  milesfor longtable footnotemark footnotetext sortkey inputdelivs bbx:url
% LocalWords:  makeatletter makeatother italicised tikzpicture paperlist jpro cht@stretch
% LocalWords:  textcolor namedef namedef nameuse setlength tabcolsep clange ppl urldate
% LocalWords:  arraystretch thedeliverable biboldfont bibfont sitename darkblue 5,step
% LocalWords:  wpsubsection wpsubsection wpsubsectionfalse wpsubsectiontrue ppl 2pt,color
% LocalWords:  definecolor medskip fundsuntil fundsuntil defernumbers shortname 2,title
% LocalWords:  totalduration circ orga notype allpapers allpaperstrue prl sym bbx:eprint
% LocalWords:  allpapersfalse subbibliography unclass inbook confpapers wphases
% LocalWords:  inproceedings wspapers notkeyword techreports techreport emphbox
% LocalWords:  textbackslash wpsectionheadings compactht compactht emphbox sym
% LocalWords:  dfgproposal Tsanko Tsankov mdframed newmdenv wphase thealltasks
% LocalWords:  alltasks taskreflong definecolorset leadgray wagray thewpfig fancyhfoffset
% LocalWords:  wpfigstyle colspan keyvals gantttaskchart baselinestretch NeedsTeXFormat
% LocalWords:  mileref mileref miletref backref firstinits wissenschaftlicher usebibmacro
%  LocalWords:  wissenschaftliche startend@wphases addtocategory localdelivref

% \endinput
% Local Variables:
% mode: doctex
% TeX-master: t
% End:
%  LocalWords:  localization sec:user-interface standardizes initialization referencable
%  LocalWords:  pages,start,length,caption sec:implwptable draft,yscale organization
%  LocalWords:  tab:collaboration sec:limitations sec:implementation ifgrantagreement
%  LocalWords:  grantagreementfalse grantagreement grantagreementtrue a4paper,twoside
%  LocalWords:  a4paper,margin alphabetic,defernumbers true,backend bibtex,firstinits
%  LocalWords:  true,maxbibnames true,linkcolor eso-foot,today proposalnumber
%  LocalWords:  specializing sortkey,due,label,id,title,type,level,lead gantt@step,very
%  LocalWords:  gantt@xscale,yscale gantt@ymiles gantt@ymiles subbibliography,type
%  LocalWords:  subbibliography,title featured,heading renewbibmacro iffieldundef
