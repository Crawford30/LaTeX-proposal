% The main file for developing the proposal. 
% Variants with different class options are 
% - submit.tex (no draft stuff, no ednotes, no revision information)
% - public.tex (like submit.tex, but no financials either) 
\providecommand{\classoptions}{,keys} % to be overwritten in variants
\documentclass[noworkareas,deliverables,gitinfo,report\classoptions]{euproposal}
\usepackage{nimbusserif}
\addbibresource{../lib/dummy.bib}
\WAperson[id=miko, 
           personaltitle=Prof. Dr.,
           birthdate=13. September 1964,
           academictitle=Professor of Computer Science,
           affiliation=jacu,
           department=case,
           privaddress=None of your business,
           privtel=that neither,
           email=m.kohlhase@jacobs-university.de,
           workaddress={Campus Ring 1, 28757 Bremen},
           worktel=+49 421 200 3140,
           worktelfax=+49 421 200 3140/493140,
           workfax=+49 421 200 493140]
           {Michael Kohlhase}

\WAperson[id=gc,
           personaltitle=Dr.,
           academictitle=Senior Researcher,
           birthdate=14. April 1972,
           affiliation=pcg,
           department=pcsa,
           privaddress=None of your business,
           privtel=that neither,
           workaddress={PCG Way 7, Hooville},
           worktel=+49 421 0815 4711,
           workfax=+49 421 0815 4712,
           email=gc@pcg.phony]
           {Great Communicator}

\WAinstitution[id=case,acronym=CASE,shortname=CASE,
                url=http://jacobs-university.de/ses/case,
                partof=jacu]
               {Center for Advanced Systems Engineering}

\WAinstitution[id=jacu,acronym=JacU,
               url=http://jacobs-university.de,
               streetaddress={Campus Ring 1},
               townzip={28759 Bremen},
               countryshort=D,
               country=Germany,
               type=University,
               logo=jacobs-logo.png,
               shortname=Jacobs University]
               {Jacobs University Bremen}

\WAinstitution[id=pcsa,
                           url=http://pcg.phony/sa,
                           partof=pcg,shortname=Science Affairs]
               {Science Affairs}
\WAinstitution[id=pcg,acronym=PCG,
                           url=http://pcg.phony,
                           countryshort=D,
                           streetaddress={Seefahrtstrasse 5},
                           townzip={23555 Hamburg},
                           shortname=Power Consulting]
               {Power Consulting GmbH}

%%% Local Variables: 
%%% mode: latex
%%% End: 

% LocalWords:  WAperson miko personaltitle academictitle privaddress privtel
% LocalWords:  workaddress worktel workfax gc worktelfax pcg
% Some sections of the included files depend on this.

\begin{document}
\begin{center}\color{red}\huge
  This mock proposal is just an example for \texttt{euproposal.cls} it reflects the ICT
  template of January 2012
\end{center}
\begin{proposal}[site=jacu,jacuRM=36,
  site=efo,efoRM=36,
  site=bar,barRM=36,
  site=baz,bazRM=36,
  coordinator=miko,
  acronym={iPoWr},
  acrolong={\underline{I}ntelligent} {\underline{P}r\underline{o}sal} {\underline{Wr}iting},
  title=\pn: \protect\pnlong,
  callname = ICT Call 1,
  callid = FP7-???-200?-?,
  instrument= Small or Medium-Scale Focused Research Project (STREP), 
  challengeid = 4,
  challenge = ICT for EU Proposals,
  objectiveid={ICT-2012.4.4}, 
  objective = Technology-enhanced Documents,
  outcomeid = b1,
  outcome = {More time for Research, not Proposal writing},
  coordinator=miko,
  months=24,
  compactht]
\begin{abstract}
  Writing grant proposals is a collaborative effort that requires the integration of
  contributions from many individuals. The use of an ASCII-based format like {\LaTeX}
  allows to coordinate the process via a source code control system like
  {\textsc{Subversion}}, allowing the proposal writing team to concentrate on the contents
  rather than the mechanics of wrangling with text fragments and revisions.
\end{abstract}

\tableofcontents

\begin{todo}{from the proposal template}
  Recommended length for the whole part B: 50--60 pages (including tables, references,
  etc.)
\end{todo}
\chapter{Scientific and Technical Quality}\label{chap:quality}
\begin{todo}{from the proposal template}
  Maximum length for the whole of Section 1 –-- twenty pages, not including the tables in
  Section 1.3
\end{todo}

\section{Concept and Objectives}\label{sec:objectives}
\begin{todo}{from the proposal template}
  Explain the concept of your project. What are the main ideas that led you to propose
  this work?  Describe in detail the S\&T objectives. Show how they relate to the topics
  addressed by the call. The objectives should be those achievable within the project, not
  through subsequent development. They should be stated in a measurable and verifiable
  form, including through the milestones that will be indicated under Section 1.3 below.
\end{todo}

%%% Local Variables: 
%%% mode: latex
%%% TeX-master: "propB"
%%% End: 

\section{Progress beyond the State-of-the-Art}\label{sec:progress}
\begin{todo}{from the proposal template}
 Describe the state-of-the-art in the area concerned, and the advance that the proposed
  project would bring about. If applicable, refer to the results of any patent search you
  might have carried out.
\end{todo}

%%% Local Variables: 
%%% mode: latex
%%% TeX-master: "propB"
%%% End: 

\section{Scientific/Technical Methodology and Work Plan}\label{sec:methodology}
\begin{todo}{from the proposal template}
  A detailed work plan should be presented, broken down into work packages\footnote{A work
    package is a major sub-division of the proposed project with a verifiable end-point –
    normally a deliverable or an important milestone in the overall project.} (WPs) which
  should follow the logical phases of the implementation of the project, and include
  consortium management and assessment of progress and results. (Note that your overall
  approach to management will be described later, in Section 2).

Notes: The number of work packages used must be appropriate to the complexity of the work
and the overall value of the proposed project. The planning should be sufficiently
detailed to justify the proposed effort and allow progress monitoring by the Commission.

Any significant risks should be identified, and contingency plans described
\end{todo}
\newpage\section{Objectives and Work Programme}\label{sec:workplan}

\subsection{Anticipated Total Duration of the Project}\label{sec:duration}

\begin{todo}{from the proposal template}
Please state
\begin{itemize}
 \item the project's intended duration 1 and how long DFG funds will be necessary,
 \item for ongoing projects: since when the project has been active.
\end{itemize}
\end{todo}

\subsection{Objectives}\label{sec:objectives}

\begin{objective}[id=firstobj,title=Supporting Authors]
  This is the first objective, after all we have to write proposals all the time, and we
  would rather spend time on research. 
\end{objective}

\begin{objective}[id=secondobj,title=Supporting Reviewers]
  They are only human too, so let's have a heart for them as well. 
\end{objective}


\subsection{Work Programme Including Proposed Research Methods}\label{sec:wawp}

%%%%%%%%%%%%%%%%%%%%%%%%%%%%%%%%%%%%%%%%%%%%%%%%%%%%%%%%%%%%%%%%%%%%%%%%%%%%%%%%%
\LaTeX is the best document markup language, it can even be used for literate
programming~\cite{DK:LP,Lamport:ladps94,Knuth:ttb84}

\begin{todo}{from the proposal template}
 review the state of the art in the and your own contribution to it; probably you want to
  divide this into subsubsections. 
\end{todo}

\begin{todo}{from the proposal template}
For each applicant

Please give a detailed account of the steps planned during the proposed funding pe-
riod. (For experimental projects, a schedule detailing all planned experiments should
be provided.)

The quality of the work programme is critical to the success of a funding proposal. The
work programme should clearly state how much funding will be requested, why the
funds are needed, and how they will be used, providing details on individual items
where applicable.

Please provide a detailed description of the methods that you plan to use in the project:
What methods are already available? What methods need to be developed? What as-
sistance is needed from outside your own group/institute?
Please list all cited publications pertaining to the description of your work programme
in your bibliography under section 3.
\end{todo}

The project is organized around \pdatacount{all}{wa} large-scale work areas which correspond
to the objectives formulated above. These are subdivided into \pdatacount{all}{wp} work
packages, which we summarize in Figure~\ref{fig:wplist}. Work area
\WAref{mansubsus} will run over the whole project\ednote{come up with a better
  example, this is still oriented towards an EU project} duration of {\pn}. All
{\pdatacount{systems}{wp}} work packages in {\WAref{systems}} will and have to be
covered simultaneously in order to benefit from design-implementation-application feedback
loops.

\wpfig

\begin{workplan}
\begin{workarea}[id=mansubsus,title={Management, Support \& Sustainability}, short=Management]
  This work-group corresponds to Objective \OBJref{firstobj} and has two work packages:
  one for management proper ({\WPref{management}}), and one each for
  dissemination ({\WPref{dissem}})
   
  This work group ensures the dissemination and creation of the periodic integrative
  reports containing the periodic Project Management Report, the Project Management
  Handbook, an Knowledge Dissemination Plan ({\WPref{management}}), the Proceedings of the
  Annual {\pn} Summer School as well as non-public Dissemination and Exploitation plans
  ({\WPref{dissem}}), as well as a report of the {\pn} project milestones.
   
\begin{workpackage}[id=management,lead=jacu,
  title=Project Management,
 jacuRM=2,jacuRAM=8,pcgRM=2]
  Based on the ``Bewilligungsbescheid'' of the DFG, and based on the financial and
  administrative data agreed, the project manager will carry out the overall project
  management, including administrative management.  A project quality handbook will be
  defined, and a {\pn} help-desk for answering questions about the format (first
  project-internal, and after month 12 public) will be established. The project management
  will consist of the following tasks
\begin{tasklist} 
\begin{task}[id=foo,wphases=0-3,requires=\taskin{t1}{dissem}]
  To perform the administrative, scientific/technical, and financial management of the
  project 
\end{task}
\begin{task}[wphases=13-17!.5]
  To co-ordinate the contacts with the DFG and other funding bodies, building on the
  results in \taskref{management}{foo}
\end{task}
\begin{task}
  To control quality and timing of project results and to resolve conflicts
\end{task}
\begin{task}
  To set up inter-project communication rules and mechanisms
\end{task}
\end{tasklist}

\end{workpackage}
 
\begin{workpackage}[id=dissem,lead=pcg,
 title=Dissemination and Exploitation,
pcgRM=8,jacuRAM=2] 
Much of the activity of a project involves small groups of nodes in joint work. This work
 package is set up to ensure their best wide-scale integration, communication, and
 synergetic presentation of the results. Clearly identified means of dissemination of
 work-in-progress as well as final results will serve the effectiveness of work within the
 project and steadily improve the visibility and usage of the emerging semantic services.


 The work package members set up events for dissemination of the research and
 work-in-progress results for researchers (workshops and summer schools), and for industry
 (trade fairs). An in-depth evaluation will be undertaken of the response of test-users.
 
 \begin{tasklist}
  \begin{task}[id=t1,wphases=6-7]
    sdfkj
  \end{task}
  \begin{task}[wphases=12-13]
    sdflkjsdf
  \end{task}
  \begin{task}[wphases=18-19]
    sdflkjsdf
  \end{task}
 \begin{task}[wphases=22-24] 
 \end{task}
\end{tasklist}

Within two months of the start of the project, a project website will go live. This
website will have two areas: a members' area and a public area.\ldots
\end{workpackage}
\end{workarea}
 

\begin{workarea}[id=systems,title={System Development}]
  This workarea does not correspond to \OBJtref{secondobj}, but it has two work packages:
  one for the development of the {\LaTeX} class ({\WPref{class}}), and for the
  proposal template ({\WPref{temple}})

  This work group coordinates the system development.

\begin{workpackage}[id=class,lead=jacu,
                    title=A LaTeX class for EU Proposals,short=Class,
                   jacuRM=12,jacuRAM=8,pcgRM=12,pcgRAM=2]
We plan to develop a {\LaTeX} class for marking up EU Proposals

We will follow strict software design principles, first comes a
requirements analys, then \ldots
\begin{tasklist}
  \begin{task}[wphases=0-2]
    sdfsdf
  \end{task}
  \begin{task}[wphases=4-8]
    sdfsdf
  \end{task}
  \begin{task}[id=t3,wphases=10-14]
    sdfsdf
  \end{task}
  \begin{task}[wphases=20-24]
    sdfsdfd
  \end{task}
\end{tasklist}
\end{workpackage} 

\begin{workpackage}[id=temple,lead=pcg,
  title= Proposal Template,short=Template,jacuRM=12]

We plan to develop a template file for {\pn} proposals

We abstract an example from existing proposals
\begin{tasklist}
  \begin{task}[wphases=6-12]
    sdfdsf 
  \end{task}
  \begin{task}[id=temple2,wphases=18-24,requires=\taskin{t3}{class}]
    sdfsdf
  \end{task} 
\end{tasklist}
\end{workpackage}

\begin{workpackage}[id=workphase,title=A work package without tasks,
  wphases=0-4!.5]
  
  And finally, a work package without tasks, so we can see the effect on the gantt chart
  in fig~\ref{fig:gantt}.
\end{workpackage}
\end{workarea}
\end{workplan} 

\ganttchart[draft,xscale=.45] 

\subsection{Data Handling}\label{sec:data}

The \pn project will not systematically produce researchdata. All project results will be
published for at least $x$ years at our archive at \url{http://example.org}.

\subsection{-- 2.7 (Other Information / Explanations on the Proposed Investigations / Information on Scientific and Financial Involvement of International
  Cooperation Partners) \qquad \sf n/a}


%%% Local Variables: 
%%% mode: LaTeX
%%% TeX-master: "proposal"
%%% End: 

% LocalWords:  workplan.tex wplist dfgcount wa mansubsus duratio ipower wpfig
% LocalWords:  ganttchart xscale workplan workarea pdataref dissem workpackage foo
% LocalWords:  tasklist taskin taskref sdfkj sdflkjsdf sdfsdf sdfsdfd sdfdsf pn
% LocalWords:  firstobj secondobj pdatacount WAref ednote OBJref pcgRM pcg
% LocalWords:  ldots OBJtref workphase


\newpage
\subsection{Work Package List}\label{sec:wplist}

\begin{todo}{from the proposal template}
Please indicate one activity per work package:
RTD = Research and technological development; DEM = Demonstration; MGT = Management of the consortium
\end{todo}

%\makeatletter\wp@total@RM{management}\makeatother
\wpfigstyle{\footnotesize}
\wpfig[pages,type,start,end]

\newpage\subsection{List of Deliverables}\label{sec:deliverables}

\begin{todo}{from the proposal template}
\begin{compactenum}
\item Deliverable numbers in order of delivery dates. Please use the numbering convention <WP number>.<number of deliverable within
that WP>. For example, deliverable 4.2 would be the second deliverable from work package 4.
\item Please indicate the nature of the deliverable using one of the following codes:
R = Report, P = Prototype, D = Demonstrator, O = Other
\item Please indicate the dissemination level using one of the following codes:
PU = Public
PP = Restricted to other programme participants (including the Commission Services).
RE = Restricted to a group specified by the consortium (including the Commission Services).
CO = Confidential, only for members of the consortium (including the Commission Services).
\end{compactenum}
\end{todo}
We will now give an overview over the deliverables and milestones of the work
packages. Note that the times of deliverables after month 24 are estimates and may change
as the work packages progress.

In the table below, {\emph{integrating work deliverables}} (see top of
section~\ref{sec:wplist}) are printed in boldface to mark them. They integrate
contributions from multiple work packages. \ednote{CL: the rest of this paragraph does not
  comply with the EU guide for applicants, needs to be rewritten}These can have the
dissemination level ``partial'', which indicates that it contains parts of level
``project'' that are to be disseminated to the project and evaluators only. In such
reports, two versions are prepared, and disseminated accordingly.

{\footnotesize\inputdelivs{8cm}}


%%% Local Variables: 
%%% mode: latex
%%% TeX-master: "propB"
%%% End: 

\newpage\subsection{List of Milestones}\label{sec:milestones}

\begin{todo}{from the proposal template}
  Milestones are control points where decisions are needed with regard to the next stage
  of the project. For example, a milestone may occur when a major result has been
  achieved, if its successful attainment is a requirement for the next phase of
  work. Another example would be a point when the consortium must decide which of several
  technologies to adopt for further development.

  Means of verification: Show how you will confirm that the milestone has been
  attained. Refer to indicators if appropriate. For examples: a laboratory prototype
  completed and running flawlessly, software released and validated by a user group, field
  survey complete and data quality validated.
\end{todo}


The work in the {\pn} project is structured by seven milestones, which coincide with the
project meetings in summer and fall.  Since the meetings are the main face-to-face
interaction points in the project, it is suitable to schedule the milestones for these
events, where they can be discussed in detail. We envision that this setup will give the
project the vital coherence in spite of the broad mix of disciplinary backgrounds of the
participants.\ednote{maybe automate the milestones}

\begin{milestones}
  \milestone[id=kickoff,verif=Inspection,month=1]
    {Initial Infrastructure}
    {Set up the organizational infrastructure, in particular: Web Presence, project TRAC,\ldots}
  \milestone[id=consensus,verif=Inspection,month=24]{Consensus} {Reach Consensus on the
    way the project goes}
  \milestone[id=exploitation,verif=Inspection,month=36]{Exploitation}{The exploitation
    plan should be clear so that we can start on this in the last year.}
  \milestone[id=final,verif=Inspection,month=48]{Final Results}{all is done}
\end{milestones}

%%% Local Variables: 
%%% mode: latex
%%% TeX-master: "propB"
%%% End: 

% LocalWords:  pn ednote verif ldots


\subsection{Work Package Descriptions}\label{sec:workpackages}
\begin{workplan}
\begin{workpackage}[id=management,type=MGT,wphases=0-24!.2,
  title=Project Management,short=Management,
  jacuRM=2,barRM=2,efoRM=2,bazRM=2,lead=jacu,status=canceled]
We can state the state of the art and similar things before the summary in the boxes
here. 
\wpheadertable
\begin{wpobjectives}
  \begin{itemize}
    \item To perform the administrative, scientific/technical, and financial
      management of the project
    \item To co-ordinate the contacts with the EU
    \item To control quality and timing of project results and to resolve conflicts
    \item To set up inter-project communication rules and mechanisms
  \end{itemize}
\end{wpobjectives}

\begin{wpdescription}
  Based on the Consortium Agreement, i.e. the contract with the European Commission, and
  based on the financial and administrative data agreed, the project manager will carry
  out the overall project management, including administrative management.  A project
  quality handbook will be defined, and a {\pn} help-desk for answering questions about
  the format (first project-internal, and after month 12 public) will be established. The
  project management will\ldots we can even reference deliverables:
  \delivref{management}{report2} and even the variant with a title:
  \delivtref{management}{report2}
\begin{tasklist}
  \begin{task}
    To perform the administrative, scientific/technical, and financial management of the
    project
    \end{task}
    \begin{task}
      To co-ordinate the contacts with the EU
    \end{task}
    \begin{task}
      To control quality and timing of project results and to resolve conflicts
    \end{task}
    \begin{task}[status=canceled]
      To set up inter-project communication rules and mechanisms
    \end{task}
  \end{tasklist}
\end{wpdescription}


\begin{wpdelivs}
  \begin{wpdeliv}[due=1,id=mailing,nature=O,dissem=PP,miles=kickoff,lead=jacu]
    {Project-internal mailing lists}
  \end{wpdeliv}
  \begin{wpdeliv}[due=3,id=handbook,nature=R,dissem=PU,miles=consensus,lead=jacu]
    {Project management handbook}
  \end{wpdeliv}
  \begin{wpdeliv}[due=44,id=worldpeace,nature=R,dissem=PU,status=canceled,lead=jacu]
    {Plan to save the world}
  \end{wpdeliv}
\begin{wpdeliv}[due={6,12,18,24,30,36,42,48},id=report2,nature=R,dissem=public,miles={consensus,final}]
  {Periodic activity report} 
  Partly compiled from activity reports of the work package
  coordinators; to be approved by the work package coordinators before delivery to the
  Commission.  Financial reporting is mainly done in months 18 and 36.\Ednote{how about
    these numbers?}
  \end{wpdeliv}
 \begin{wpdeliv}[due=6,id=helpdesk,dissem=PU,nature=O,miles=kickoff,lead=jacu]
    {{\pn} Helpdesk}
  \end{wpdeliv}
  \begin{wpdeliv}[due=36,id=report6,nature=R,dissem=PU,miles=final,lead=jacu]
    {Final plan for using and disseminating the knowledge}
  \end{wpdeliv}
  \begin{wpdeliv}[due=48,id=report7,nature=R,dissem=PU,miles=final,lead=jacu]
    {Final management report}
  \end{wpdeliv}
\end{wpdelivs}
\end{workpackage}

%%% Local Variables: 
%%% mode: LaTeX
%%% TeX-master: "propB"
%%% End: 

% LocalWords:  wp-management.tex workpackage efoRM bazRM wpheadertable pn ldots
% LocalWords:  wpobjectives wpdescription delivref delivtref wpdelivs wpdeliv
% LocalWords:  dissem Ednote pdataRef deliv mansubsusintReport wphases
\newpage
\begin{workpackage}%
[id=dissem,type=RTD,lead=efo,
 wphases=10-24!1,
 title=Dissemination and Exploitation,short=Dissemination,
 efoRM=8,jacuRM=2,barRM=2,bazRM=2]
We can state the state of the art and similar things before the summary in the boxes
here. 
\wpheadertable

\begin{wpobjectives}
  Much of the activity of a project involves small groups of nodes in joint work. This
  work package is set up to ensure their best wide-scale integration, communication, and
  synergetic presentation of the results. Clearly identified means of dissemination of
  work-in-progress as well as final results will serve the effectiveness of work within
  the project and steadily improve the visibility and usage of the emerging semantic
  services.
\end{wpobjectives}

\begin{wpdescription}
  The work package members set up events for dissemination of the research and
  work-in-progress results for researchers (workshops and summer schools), and for
  industry (trade fairs). An in-depth evaluation will be undertaken of the response of
  test-users.

  Within two months of the start of the project, a project website will go live. This
  website will have two areas: a members' area and a public area.\ldots
\end{wpdescription}

\begin{wpdelivs}
  \begin{wpdeliv}[due=2,id=website,nature=O,dissem=PU,miles=kickoff]
     {Set-up of the Project web server}
   \end{wpdeliv}
   \begin{wpdeliv}[due=8,id=ws1proc,nature=R,dissem=PU,miles={kickoff}]
     {Proceedings of the first {\pn} Summer School.}
   \end{wpdeliv}
   \begin{wpdeliv}[due=9,id=dissem,nature=R,dissem=PP]
     {Dissemination Plan}
   \end{wpdeliv}
   \begin{wpdeliv}[due=9,id=exploitplan,nature=R,dissem=PP,miles=exploitation]
     {Scientific and Commercial Exploitation Plan}
   \end{wpdeliv}
   \begin{wpdeliv}[due=20,id=ws2proc,nature=R,dissem=PU,miles={exploitation}]
     {Proceedings of the second {\pn} Summer School.}
   \end{wpdeliv}
   \begin{wpdeliv}[due=32,id=ss1proc,nature=R,dissem=PU,miles={exploitation}]
     {Proceedings of the third {\pn} Summer School.}
   \end{wpdeliv}
   \begin{wpdeliv}[due=44,id=ws3proc,nature=R,dissem=PU,miles=exploitation]
     {Proceedings of the fourth {\pn} Summer School.}
   \end{wpdeliv}
 \end{wpdelivs}
\end{workpackage}

%%% Local Variables: 
%%% mode: LaTeX
%%% TeX-master: "propB"
%%% End: 

% LocalWords:  wp-dissem.tex workpackage dissem efo fromto bazRM wpheadertable
% LocalWords:  wpobjectives wpdescription ldots wpdelivs wpdeliv ws1proc pn
% LocalWords:  exploitplan ws2proc ss1proc ws3proc pdataRef deliv
% LocalWords:  mansubsusintReport
\newpage
\begin{workpackage}[id=class,type=RTD,lead=jacu,
                    wphases=3-9!1,
                    title=A {\LaTeX} class for EU Proposals,short=Class,
                    jacuRM=12,barRM=12]
We can state the state of the art and similar things before the summary in the boxes
here. 
\wpheadertable
\begin{wpobjectives}
\LaTeX is the best document markup language, it can even be used for literate
programming~\cite{DK:LP,Lamport:ladps94,Knuth:ttb84}

  To develop a {\LaTeX} class for marking up EU Proposals
\end{wpobjectives}

\begin{wpdescription}
  We will follow strict software design principles, first comes a requirements analys,
  then \ldots
\end{wpdescription}

\begin{wpdelivs}
  \begin{wpdeliv}[due=6,id=req,nature=R,dissem=PP,miles=kickoff]
     {Requirements analysis}
   \end{wpdeliv}
   \begin{wpdeliv}[due=12,id=spec,nature=R,dissem=PU,miles=consensus]
     {{\pn} Specification }
   \end{wpdeliv}
   \begin{wpdeliv}[due=18,id=demonstrator,nature=P,dissem=PU,miles={consensus,final}]
     {First demonstrator ({\tt{article.cls}} really)}
   \end{wpdeliv}
   \begin{wpdeliv}[due=24,id=proto,nature=P,dissem=PU,miles=final]
     {First prototype}
   \end{wpdeliv}
    \begin{wpdeliv}[due=36,id=release,nature=P,dissem=PU,miles=final]
      {Final {\LaTeX} class, ready for release}
    \end{wpdeliv}
  \end{wpdelivs}
Furthermore, this work package contributes to {\pdataRef{deliv}{managementreport2}{label}} and
{\pdataRef{deliv}{managementreport7}{label}}.
\end{workpackage}

%%% Local Variables: 
%%% mode: LaTeX
%%% TeX-master: "propB"
%%% End: 
\newpage
\begin{workpackage}[id=temple,type=DEM,lead=bar,
  wphases=6-12!1,
  title={\pn} Proposal Template,short=Template,barRM=6,bazRM=6]
We can state the state of the art and similar things before the summary in the boxes
here. 
\wpheadertable

\begin{wpobjectives}
  To develop a template file for {\pn} proposals
\end{wpobjectives}

\begin{wpdescription}
  We abstract an example from existing proposals
\end{wpdescription}

\begin{wpdelivs}
  \begin{wpdeliv}[due=6,id=req,nature=R,dissem=PP,miles=kickoff,lead=bar]
    {Requirements analysis}
  \end{wpdeliv}
  \begin{wpdeliv}[due=12,id=spec,nature=R,dissem=PU,miles=consensus,lead=baz]
    {{\pn} Specification }
  \end{wpdeliv}
  \begin{wpdeliv}[due=18,id=demonstrator,nature=D,dissem=PU,miles={consensus,final},lead=bar]
    {First demonstrator ({\tt{article.cls}} really)}
  \end{wpdeliv}
  \begin{wpdeliv}[due=24,id=proto,nature=P,dissem=PU,miles=final,lead=baz]
    {First prototype}
  \end{wpdeliv}
  \begin{wpdeliv}[due=36,id=release,nature=P,dissem=PU,miles=final,lead=bar]
    {Final Template, ready for release}
  \end{wpdeliv}
\end{wpdelivs}
Furthermore, this work package contributes to {\pdataRef{deliv}{managementreport2}{label}} and
{\pdataRef{deliv}{managementreport7}{label}}.
\end{workpackage}

%%% Local Variables: 
%%% mode: LaTeX
%%% TeX-master: "propB"
%%% End: 

% LocalWords:  wp-temple.tex workpackage fromto pn bazRM wpheadertable wpdelivs
% LocalWords:  wpobjectives wpdescription wpdeliv req dissem tt article.cls
% LocalWords:  pdataRef deliv systemsintReport
\newpage
\end{workplan}
\newpage\subsection{Significant Risks and Associated Contingency Plans}\label{sec:risks}
\begin{todo}{from the proposal template}
  Describe any significant risks, and associated contingency plans
\end{todo}
\begin{oldpart}{need to integrate this somewhere. CL: I will check other proposals to see how they did it; the Guide does not really prescribe anything.}
\paragraph{Global Risk Management}
The crucial problem of \pn (and similar endeavors that offer a new basis for communication
and interaction) is that of community uptake: Unless we can convince scientists and
knowledge workers industry to use the new tools and interactions, we will
never be able to assemble the large repositories of flexiformal mathematical knowledge we
envision. We will consider uptake to be the main ongoing evaluation criterion for the network.
\end{oldpart}

%%% Local Variables: 
%%% mode: latex
%%% TeX-master: "propB"
%%% End: 



%%% Local Variables: 
%%% mode: latex
%%% TeX-master: "propB"
%%% End: 

% LocalWords:  workplan newpage wplist makeatletter makeatother wpfig
% LocalWords:  workpackages wp-dissem wp-class wp-temple

%%% Local Variables: 
%%% mode: LaTeX
%%% TeX-master: "propB"
%%% End: 

\newpage
\chapter{Implementation}\label{chap:implementation}

\section{Management Structure and Procedures}\label{chap:management}
\begin{todo}{from the proposal template}
  Describe the organizational structure and decision-making mechanisms
  of the project. Show how they are matched to the nature, complexity
  and scale of the project.  Maximum length of this section: five pages.
\end{todo}

The Project Management of {\pn} is based on its Consortium Agreement, which will be
signed before the Contract is signed by the Commission. The Consortium Agreement will
enter into force as from the date the contract with the European Commission is signed.
\subsection{Organizational structure}\label{sec:management-structure}
\subsection{Milestones}\label{sec:milestones}
\milestonetable
\subsection{Risk Assessment and Management}
\subsection{Information Flow and Outreach}\label{sec:spread-excellence}
\subsection{Quality Procedures}\label{sec:quality-management}
\subsection{Internal Evaluation Procedures}
\newpage

\section{The {\protect\pn} consortium as a whole}
\begin{todo}{from the proposal template}
  Describe how the participants collectively constitute a consortium capable of achieving
  the project objectives, and how they are suited and are committed to the tasks assigned
  to them. Show the complementarity between participants. Explain how the composition of
  the consortium is well-balanced in relation to the objectives of the project.  

  If appropriate describe the industrial/commercial involvement to ensure exploitation of
  the results. Show how the opportunity of involving SMEs has been addressed
\end{todo}

The project partners of the \pn project have a long history of successful collaboration;
Figure~\ref{tab:collaboration} gives an overview over joint projects (including proposals) and
joint publications (only international, peer reviewed ones).

\jointorga{jacu,efo,baz}
\jointpub{efo,baz,jacu}
\jointproj{efo,bar}
\jointsup{jacu,bar}
\jointsoft{baz,efo}
\coherencetable

\subsection{Subcontracting}\label{sec:subcontracting}
\begin{todo}{from the proposal template}
  If any part of the work is to be sub-contracted by the participant responsible for it,
  describe the work involved and explain why a sub-contract approach has been chosen for
  it.
\end{todo}
\subsection{Other Countries}\label{sec:other-countries}
\begin{todo}{from the proposal template}
  If a one or more of the participants requesting EU funding is based outside of the EU
  Member states, Associated countries and the list of International Cooperation Partner
  Countries\footnote{See CORDIS web-site, and annex 1 of the work programme.}, explain in
  terms of the project’s objectives why such funding would be essential.
\end{todo}

\subsection{Additional Partners}\label{sec:assoc-partner}
\begin{todo}{from the proposal template}
  If there are as-yet-unidentified participants in the project, the expected competences,
  the role of the potential participants and their integration into the running project
  should be described
\end{todo}
\section{Resources to be Committed}\label{sec:resources}
\begin{todo}{from the proposal template}
Maximum length: two pages

Describe how the totality of the necessary resources will be mobilized, including any resources that
will complement the EC contribution. Show how the resources will be integrated in a coherent way,
and show how the overall financial plan for the project is adequate.

In addition to the costs indicated on form A3 of the proposal, and the effort shown in Section 1.3
above, please identify any other major costs (e.g. equipment). Ensure that the figures stated in Part B
are consistent with these.
\end{todo}

\subsection{Travel Costs and Consumables}\label{sec:travel-costs}
\subsection{Subcontracting Costs}
\subsection{Other Costs}

%%% Local Variables: 
%%% mode: LaTeX
%%% TeX-master: "propB"
%%% End: 

% LocalWords:  pn newpage site-jacu site-efo site-baz jointpub efo baz
% LocalWords:  jointproj coherencetable assoc-partner
\newpage
\chapter{Impact}\label{chap:impact}
\ednote{Maximum length for the whole of Section 3 –-- ten pages}

\section{Expected Impacts listed in the Work Programe }\label{sec:expected-impact}
\begin{todo}{from the proposal template}
  Describe how your project will contribute towards the expected impacts listed in the
  work programme in relation to the topic or topics in question. Mention the steps that
  will be needed to bring about these impacts. Explain why this contribution requires a
  European (rather than a national or local) approach. Indicate how account is taken of
  other national or international research activities. Mention any assumptions and
  external factors that may determine whether the impacts will be achieved.
\end{todo}
\subsection{Medium Term Expected Outcome}

\subsection{Long Term Expected Outcomes}
\subsection{Use Cases}

\section{Dissemination and/or Use of Project Results, and Management of Intellectual Property}\label{sec:dissemination}

\begin{todo}{from the proposal template}
  Describe the measures you propose for the dissemination and/or exploitation of project
  results, and how these will increase the impact of the project. In designing these
  measures, you should take into account a variety of communication means and target
  groups as appropriate (e.g. policy-makers, interest groups, media and the public at
  large).

  For more information on communication guidance, see the URL
  \url{http://ec.europa.eu/research/science-society/science-communication/index_en.htm}

  Describe also your plans for the management of knowledge (intellectual property)
  acquired in the course of the project.
\end{todo}


%%% Local Variables: 
%%% mode: LaTeX
%%% TeX-master: "propB"
%%% End: 

% LocalWords:  ednote
\newpage
\end{proposal}
\section{Individual Participants}\label{sec:partners}
\begin{todo}{from the proposal template}
For each participant in the proposed project, provide a brief description of the legal entity, the main
tasks they have been attributed, and the previous experience relevant to those tasks. Provide also a
short profile of the individuals who will be undertaking the work.\\
Maximum length for Section 2.2: one page per participant. However, where two or more departments within
an organisation have quite distinct roles within the proposal, one page per department is acceptable.\\
The maximum length applying to a legal entity composed of several members, each of which is a separate
legal entity (for example an EEIG1), is one page per member, provided that the members have quite distinct
roles within the proposal.
\end{todo}
\newpage
\begin{sitedescription}{jacu}

\paragraph{Organization} Jacobs University Bremen is a private research university patterned
after the Anglo-Saxon university system.  The university opened in
2001 and has an international student body ($1,245$ students from 102
nations as of 2011, admitted in a highly selective process).

The KWARC (KnoWledge Adaptation and Reasoning for
Content\footnote{\url{http://kwarc.info}}) Group headed by
{\emph{Prof.\ Dr.\ Michael Kohlhase}} specializes in building
knowledge management systems for e-science applications, in particular
for the natural and mathematical sciences.  Formal logic, natural
language semantics, and semantic web technology provide the
foundations for the research of the group.
  
  Since doing research and developing systems is much more fun than writing proposals,
  they try go do that as efficiently as possible, hence this meta-proposal. 

\paragraph{Main tasks}

\begin{itemize}
\item creating {\LaTeX} class files
\end{itemize}

\paragraph{Relevant previous experience}

The KWARC group is the main center and lead implementor of the OMDoc
(Open Mathematical Document) format for representing mathematical
knowledge.  The group has developed added-value services powered by such semantically rich representations, different paths to obtaining them, as well as platforms that integrate both aspects.  Services include the adaptive context-sensitive presentation framework JOMDoc and the semantic search engine MathWebSearch.  For obtaining rich mathematical content, the group has been pursuing the two alternatives of assisting manual editing (with the sTeXIDE editing environment) and automatic annotation using natural language processing techniques.  The latter is work in progress but builds on the arXMLiv system, which is currently capable of converting 70\% out of the 600,000 scientific publications in the arXiv from {\LaTeX} to XHTML+MathML without errors.  Finally, the KWARC group has been developing the Planetary integrated environment.

\paragraph{Specific expertise}

\begin{itemize}
\item writing intelligent proposals
\end{itemize}

\paragraph{Staff members involved}

\textbf{Prof.\ Dr.\ Michael Kohlhase} is head of the KWARC research
group.  He is the head developer of the OMDoc mathematical markup
language.  He was a member of the Math Working Group at W3C, which finished its work with the publication of the MathML 3 recommendation.  He is president of the OpenMath society and trustee of the MKM
interest group.

\keypubs{KohDavGin:psewads11,Kohlhase:pdpl10,Kohlhase:omdoc1.2,CarlisleEd:MathML10,StaKoh:tlcspx10}
\end{sitedescription}

%%% Local Variables: 
%%% mode: LaTeX
%%% TeX-master: "propB"
%%% End: 

% LocalWords:  site-jacu.tex sitedescription emph textbf keypubs KohDavGin
% LocalWords:  psewads11 pdpl10 StaKoh tlcspx10 KohDavGin:psewads11,Kohlhase:pdpl10
\newpage
\begin{sitedescription}{efo}
\paragraph{Organization}
 The EFO is the world leader in futurology, \ldots
\paragraph{Main tasks}
\paragraph{Relevant previous experience}
\paragraph{Specific expertise}
\paragraph{Staff members undertaking the work}
\keypubs{providemore}
\end{sitedescription}

%%% Local Variables: 
%%% mode: LaTeX
%%% TeX-master: "propB"
%%% End: 
\newpage
\begin{sitedescription}{bar}

\paragraph{Organization}
  Universit\'e de BAR specializes on drinking lots of red wine. It is a partner in the
  consortium, because it has a very nice chateau on the Cote d'Azure, where it can host
  gorgeous project meetings.

\paragraph{Main tasks}
\paragraph{Relevant previous experience}
\paragraph{Specific expertise}
\paragraph{Staff members undertaking the work}
\keypubs{providemore}

\end{sitedescription}

%%% Local Variables: 
%%% mode: LaTeX
%%% TeX-master: "propB"
%%% End: 
\newpage
\begin{sitedescription}{baz}
\paragraph{Organization}
\paragraph{Main tasks}
\paragraph{Relevant previous experience}
\paragraph{Specific expertise}
\paragraph{Staff members undertaking the work}
\keypubs{providemore}
\end{sitedescription}

%%% Local Variables: 
%%% mode: LaTeX
%%% TeX-master: "propB"
%%% End: 
\newpage

%%% Local Variables:
%%% mode: latex
%%% TeX-master: "propB"
%%% End:

\chapter{Ethical Issues}\label{chap:ethical}
\begin{todo}{from the proposal template}
  Describe any ethical issues that may arise in the project. In particular, you should
  explain the benefit and burden of the experiments and the effects it may have on the
  research subject. Identify the countries where research will be undertaken and which
  ethical committees and regulatory organisations will need to be approached during the
  life of the project.

  Include the Ethical issues table below.  If you indicate YES to any issue, please
  identify the pages in the proposal where this ethical issue is described. Answering
  'YES' to some of these boxes does not automatically lead to an ethical review1.  It
  enables the independent experts to decide if an ethical review is required. If you are
  sure that none of the issues apply to your proposal, simply tick the YES box in the last
  row.
\end{todo}

\begin{small}
\begin{tabular}{|p{1em}p{11cm}|l|l|}\hline
  \multicolumn{2}{|l|}{\cellcolor{lightgray}{\strut}} & 
  \cellcolor{lightgray}{YES} & 
  \cellcolor{lightgray}{PAGE}\\\hline 
  \multicolumn{2}{|l|}{\bf{Informed Consent}} & & \\\hline
  & Does the proposal involve children?  & & \\\hline
  & Does the proposal involve patients or persons not able to give consent? & & \\\hline
  & Does the proposal involve adult healthy volunteers? & & \\\hline
  & Does the proposal involve Human Genetic Material? & & \\\hline
  & Does the proposal involve Human biological samples? & & \\\hline
  & Does the proposal involve Human data collection? & & \\\hline
  \multicolumn{2}{|l|}{\bf{Research on Human embryo/foetus}}  & & \\\hline
  & Does the proposal involve Human Embryos? & & \\\hline
  & Does the proposal involve Human Foetal Tissue / Cells? & & \\\hline
  & Does the proposal involve Human Embryonic Stem Cells? & & \\\hline
  \multicolumn{2}{|l|}{\bf{Privacy}} & & \\\hline
  & Does the proposal involve processing of genetic information 
         or personal data (eg. health, sexual lifestyle, ethnicity, 
         political opinion, religious or philosophical conviction)  & & \\\hline 
  & Does the proposal involve tracking the location or observation 
         of people? & & \\\hline 
  \multicolumn{2}{|l|}{\bf{Research on Animals}} & & \\\hline 
  & Does the proposal involve research on animals? & & \\\hline 
  & Are those animals transgenic small laboratory animals? & & \\\hline 
  & Are those animals transgenic farm animals? & & \\\hline 
  & Are those animals cloned farm animals? & & \\\hline 
  & Are those animals non-human primates?  & & \\\hline 
  \multicolumn{2}{|l|}{\bf{Research Involving Developing Countries}} & & \\\hline 
  & Use of local resources (genetic, animal, plant etc) & & \\\hline 
  & Benefit to local community (capacity building 
         i.e. access to healthcare, education etc) & & \\\hline 
  \multicolumn{2}{|l|}{\bf{Dual Use}} & & \\\hline 
  & Research having direct military application  & & \\\hline 
  & Research having the potential for terrorist abuse & & \\\hline 
  \multicolumn{2}{|l|}{\bf{ICT Implants}} & & \\\hline 
  & Does the proposal involve clinical trials of ICT implants?  & & \\\hline 
  \multicolumn{2}{|l|}{\bf\footnotesize{I CONFIRM THAT NONE OF THE ABOVE ISSUES APPLY TO MY PROPOSAL}} 
      & &\cellcolor{lightgray}{} \\\hline 
\end{tabular}
\end{small}

\section{Personal Data}

\end{document}

%%% Local Variables: 
%%% mode: LaTeX
%%% TeX-master: t
%%% End: 

% LocalWords:  efo efoRM baz bazRM miko acrolong ntelligent iting pn pnlong
% LocalWords:  textsc newpage compactht texttt euproposal.cls callname callid
% LocalWords:  challengeid objectiveid outcomeid tableofcontents
